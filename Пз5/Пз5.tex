\documentclass[11pt]{article}

    \usepackage[breakable]{tcolorbox}
    \usepackage{parskip} % Stop auto-indenting (to mimic markdown behaviour)
    
    \usepackage{fontspec}          % Дозволяє використовувати системні шрифти
    \setmainfont{Times New Roman} % Встановлює основний шрифт (переконайтеся, що він є)
    \usepackage[main=ukrainian]{babel} % Включає правила для української мови
% -----------------------------------------

    \usepackage{graphicx} % Для вставки зображень
    \usepackage{amsmath}

    \title{Практична робота №5}
    \author{Озівський Владислав}
    \date{}
    % Basic figure setup, for now with no caption control since it's done
    % automatically by Pandoc (which extracts ![](path) syntax from Markdown).
    \usepackage{graphicx}
    % Keep aspect ratio if custom image width or height is specified
    \setkeys{Gin}{keepaspectratio}
    % Maintain compatibility with old templates. Remove in nbconvert 6.0
    \let\Oldincludegraphics\includegraphics
    % Ensure that by default, figures have no caption (until we provide a
    % proper Figure object with a Caption API and a way to capture that
    % in the conversion process - todo).
    \usepackage{caption}
    \DeclareCaptionFormat{nocaption}{}
    \captionsetup{format=nocaption,aboveskip=0pt,belowskip=0pt}

    \usepackage{float}
    \floatplacement{figure}{H} % forces figures to be placed at the correct location
    \usepackage{xcolor} % Allow colors to be defined
    \usepackage{enumerate} % Needed for markdown enumerations to work
    \usepackage{geometry} % Used to adjust the document margins
    \usepackage{amsmath} % Equations
    \usepackage{amssymb} % Equations
    \usepackage{textcomp} % defines textquotesingle
    % Hack from http://tex.stackexchange.com/a/47451/13684:
    \AtBeginDocument{%
        \def\PYZsq{\textquotesingle}% Upright quotes in Pygmentized code
    }
    \usepackage{upquote} % Upright quotes for verbatim code
    \usepackage{eurosym} % defines \euro

    \usepackage{iftex}
    \ifPDFTeX
        \usepackage[T1]{fontenc}
        \IfFileExists{alphabeta.sty}{
              \usepackage{alphabeta}
          }{
              \usepackage[mathletters]{ucs}
              \usepackage[utf8x]{inputenc}
          }
    \else
        \usepackage{fontspec}
        \usepackage{unicode-math}
    \fi

    \usepackage{fancyvrb} % verbatim replacement that allows latex
    \usepackage{grffile} % extends the file name processing of package graphics
                         % to support a larger range
    \makeatletter % fix for old versions of grffile with XeLaTeX
    \@ifpackagelater{grffile}{2019/11/01}
    {
      % Do nothing on new versions
    }
    {
      \def\Gread@@xetex#1{%
        \IfFileExists{"\Gin@base".bb}%
        {\Gread@eps{\Gin@base.bb}}%
        {\Gread@@xetex@aux#1}%
      }
    }
    \makeatother
    \usepackage[Export]{adjustbox} % Used to constrain images to a maximum size
    \adjustboxset{max size={0.9\linewidth}{0.9\paperheight}}

    % The hyperref package gives us a pdf with properly built
    % internal navigation ('pdf bookmarks' for the table of contents,
    % internal cross-reference links, web links for URLs, etc.)
    \usepackage{hyperref}
    % The default LaTeX title has an obnoxious amount of whitespace. By default,
    % titling removes some of it. It also provides customization options.
    \usepackage{titling}
    \usepackage{longtable} % longtable support required by pandoc >1.10
    \usepackage{booktabs}  % table support for pandoc > 1.12.2
    \usepackage{array}     % table support for pandoc >= 2.11.3
    \usepackage{calc}      % table minipage width calculation for pandoc >= 2.11.1
    \usepackage[inline]{enumitem} % IRkernel/repr support (it uses the enumerate* environment)
    \usepackage[normalem]{ulem} % ulem is needed to support strikethroughs (\sout)
                                % normalem makes italics be italics, not underlines
    \usepackage{soul}      % strikethrough (\st) support for pandoc >= 3.0.0
    \usepackage{mathrsfs}
    

    
    % Colors for the hyperref package
    \definecolor{urlcolor}{rgb}{0,.145,.698}
    \definecolor{linkcolor}{rgb}{.71,0.21,0.01}
    \definecolor{citecolor}{rgb}{.12,.54,.11}

    % ANSI colors
    \definecolor{ansi-black}{HTML}{3E424D}
    \definecolor{ansi-black-intense}{HTML}{282C36}
    \definecolor{ansi-red}{HTML}{E75C58}
    \definecolor{ansi-red-intense}{HTML}{B22B31}
    \definecolor{ansi-green}{HTML}{00A250}
    \definecolor{ansi-green-intense}{HTML}{007427}
    \definecolor{ansi-yellow}{HTML}{DDB62B}
    \definecolor{ansi-yellow-intense}{HTML}{B27D12}
    \definecolor{ansi-blue}{HTML}{208FFB}
    \definecolor{ansi-blue-intense}{HTML}{0065CA}
    \definecolor{ansi-magenta}{HTML}{D160C4}
    \definecolor{ansi-magenta-intense}{HTML}{A03196}
    \definecolor{ansi-cyan}{HTML}{60C6C8}
    \definecolor{ansi-cyan-intense}{HTML}{258F8F}
    \definecolor{ansi-white}{HTML}{C5C1B4}
    \definecolor{ansi-white-intense}{HTML}{A1A6B2}
    \definecolor{ansi-default-inverse-fg}{HTML}{FFFFFF}
    \definecolor{ansi-default-inverse-bg}{HTML}{000000}

    % common color for the border for error outputs.
    \definecolor{outerrorbackground}{HTML}{FFDFDF}

    % commands and environments needed by pandoc snippets
    % extracted from the output of `pandoc -s`
    \providecommand{\tightlist}{%
      \setlength{\itemsep}{0pt}\setlength{\parskip}{0pt}}
    \DefineVerbatimEnvironment{Highlighting}{Verbatim}{commandchars=\\\{\}}
    % Add ',fontsize=\small' for more characters per line
    \newenvironment{Shaded}{}{}
    \newcommand{\KeywordTok}[1]{\textcolor[rgb]{0.00,0.44,0.13}{\textbf{{#1}}}}
    \newcommand{\DataTypeTok}[1]{\textcolor[rgb]{0.56,0.13,0.00}{{#1}}}
    \newcommand{\DecValTok}[1]{\textcolor[rgb]{0.25,0.63,0.44}{{#1}}}
    \newcommand{\BaseNTok}[1]{\textcolor[rgb]{0.25,0.63,0.44}{{#1}}}
    \newcommand{\FloatTok}[1]{\textcolor[rgb]{0.25,0.63,0.44}{{#1}}}
    \newcommand{\CharTok}[1]{\textcolor[rgb]{0.25,0.44,0.63}{{#1}}}
    \newcommand{\StringTok}[1]{\textcolor[rgb]{0.25,0.44,0.63}{{#1}}}
    \newcommand{\CommentTok}[1]{\textcolor[rgb]{0.38,0.63,0.69}{\textit{{#1}}}}
    \newcommand{\OtherTok}[1]{\textcolor[rgb]{0.00,0.44,0.13}{{#1}}}
    \newcommand{\AlertTok}[1]{\textcolor[rgb]{1.00,0.00,0.00}{\textbf{{#1}}}}
    \newcommand{\FunctionTok}[1]{\textcolor[rgb]{0.02,0.16,0.49}{{#1}}}
    \newcommand{\RegionMarkerTok}[1]{{#1}}
    \newcommand{\ErrorTok}[1]{\textcolor[rgb]{1.00,0.00,0.00}{\textbf{{#1}}}}
    \newcommand{\NormalTok}[1]{{#1}}

    % Additional commands for more recent versions of Pandoc
    \newcommand{\ConstantTok}[1]{\textcolor[rgb]{0.53,0.00,0.00}{{#1}}}
    \newcommand{\SpecialCharTok}[1]{\textcolor[rgb]{0.25,0.44,0.63}{{#1}}}
    \newcommand{\VerbatimStringTok}[1]{\textcolor[rgb]{0.25,0.44,0.63}{{#1}}}
    \newcommand{\SpecialStringTok}[1]{\textcolor[rgb]{0.73,0.40,0.53}{{#1}}}
    \newcommand{\ImportTok}[1]{{#1}}
    \newcommand{\DocumentationTok}[1]{\textcolor[rgb]{0.73,0.13,0.13}{\textit{{#1}}}}
    \newcommand{\AnnotationTok}[1]{\textcolor[rgb]{0.38,0.63,0.69}{\textbf{\textit{{#1}}}}}
    \newcommand{\CommentVarTok}[1]{\textcolor[rgb]{0.38,0.63,0.69}{\textbf{\textit{{#1}}}}}
    \newcommand{\VariableTok}[1]{\textcolor[rgb]{0.10,0.09,0.49}{{#1}}}
    \newcommand{\ControlFlowTok}[1]{\textcolor[rgb]{0.00,0.44,0.13}{\textbf{{#1}}}}
    \newcommand{\OperatorTok}[1]{\textcolor[rgb]{0.40,0.40,0.40}{{#1}}}
    \newcommand{\BuiltInTok}[1]{{#1}}
    \newcommand{\ExtensionTok}[1]{{#1}}
    \newcommand{\PreprocessorTok}[1]{\textcolor[rgb]{0.74,0.48,0.00}{{#1}}}
    \newcommand{\AttributeTok}[1]{\textcolor[rgb]{0.49,0.56,0.16}{{#1}}}
    \newcommand{\InformationTok}[1]{\textcolor[rgb]{0.38,0.63,0.69}{\textbf{\textit{{#1}}}}}
    \newcommand{\WarningTok}[1]{\textcolor[rgb]{0.38,0.63,0.69}{\textbf{\textit{{#1}}}}}
    \makeatletter
    \newsavebox\pandoc@box
    \newcommand*\pandocbounded[1]{%
      \sbox\pandoc@box{#1}%
      % scaling factors for width and height
      \Gscale@div\@tempa\textheight{\dimexpr\ht\pandoc@box+\dp\pandoc@box\relax}%
      \Gscale@div\@tempb\linewidth{\wd\pandoc@box}%
      % select the smaller of both
      \ifdim\@tempb\p@<\@tempa\p@
        \let\@tempa\@tempb
      \fi
      % scaling accordingly (\@tempa < 1)
      \ifdim\@tempa\p@<\p@
        \scalebox{\@tempa}{\usebox\pandoc@box}%
      % scaling not needed, use as it is
      \else
        \usebox{\pandoc@box}%
      \fi
    }
    \makeatother

    % Define a nice break command that doesn't care if a line doesn't already
    % exist.
    \def\br{\hspace*{\fill} \\* }
    % Math Jax compatibility definitions
    \def\gt{>}
    \def\lt{<}
    \let\Oldtex\TeX
    \let\Oldlatex\LaTeX
    \renewcommand{\TeX}{\textrm{\Oldtex}}
    \renewcommand{\LaTeX}{\textrm{\Oldlatex}}
    % Document parameters
    % Document title
    
    
    
    
    
    
    
% Pygments definitions
\makeatletter
\def\PY@reset{\let\PY@it=\relax \let\PY@bf=\relax%
    \let\PY@ul=\relax \let\PY@tc=\relax%
    \let\PY@bc=\relax \let\PY@ff=\relax}
\def\PY@tok#1{\csname PY@tok@#1\endcsname}
\def\PY@toks#1+{\ifx\relax#1\empty\else%
    \PY@tok{#1}\expandafter\PY@toks\fi}
\def\PY@do#1{\PY@bc{\PY@tc{\PY@ul{%
    \PY@it{\PY@bf{\PY@ff{#1}}}}}}}
\def\PY#1#2{\PY@reset\PY@toks#1+\relax+\PY@do{#2}}

\@namedef{PY@tok@w}{\def\PY@tc##1{\textcolor[rgb]{0.73,0.73,0.73}{##1}}}
\@namedef{PY@tok@c}{\let\PY@it=\textit\def\PY@tc##1{\textcolor[rgb]{0.24,0.48,0.48}{##1}}}
\@namedef{PY@tok@cp}{\def\PY@tc##1{\textcolor[rgb]{0.61,0.40,0.00}{##1}}}
\@namedef{PY@tok@k}{\let\PY@bf=\textbf\def\PY@tc##1{\textcolor[rgb]{0.00,0.50,0.00}{##1}}}
\@namedef{PY@tok@kp}{\def\PY@tc##1{\textcolor[rgb]{0.00,0.50,0.00}{##1}}}
\@namedef{PY@tok@kt}{\def\PY@tc##1{\textcolor[rgb]{0.69,0.00,0.25}{##1}}}
\@namedef{PY@tok@o}{\def\PY@tc##1{\textcolor[rgb]{0.40,0.40,0.40}{##1}}}
\@namedef{PY@tok@ow}{\let\PY@bf=\textbf\def\PY@tc##1{\textcolor[rgb]{0.67,0.13,1.00}{##1}}}
\@namedef{PY@tok@nb}{\def\PY@tc##1{\textcolor[rgb]{0.00,0.50,0.00}{##1}}}
\@namedef{PY@tok@nf}{\def\PY@tc##1{\textcolor[rgb]{0.00,0.00,1.00}{##1}}}
\@namedef{PY@tok@nc}{\let\PY@bf=\textbf\def\PY@tc##1{\textcolor[rgb]{0.00,0.00,1.00}{##1}}}
\@namedef{PY@tok@nn}{\let\PY@bf=\textbf\def\PY@tc##1{\textcolor[rgb]{0.00,0.00,1.00}{##1}}}
\@namedef{PY@tok@ne}{\let\PY@bf=\textbf\def\PY@tc##1{\textcolor[rgb]{0.80,0.25,0.22}{##1}}}
\@namedef{PY@tok@nv}{\def\PY@tc##1{\textcolor[rgb]{0.10,0.09,0.49}{##1}}}
\@namedef{PY@tok@no}{\def\PY@tc##1{\textcolor[rgb]{0.53,0.00,0.00}{##1}}}
\@namedef{PY@tok@nl}{\def\PY@tc##1{\textcolor[rgb]{0.46,0.46,0.00}{##1}}}
\@namedef{PY@tok@ni}{\let\PY@bf=\textbf\def\PY@tc##1{\textcolor[rgb]{0.44,0.44,0.44}{##1}}}
\@namedef{PY@tok@na}{\def\PY@tc##1{\textcolor[rgb]{0.41,0.47,0.13}{##1}}}
\@namedef{PY@tok@nt}{\let\PY@bf=\textbf\def\PY@tc##1{\textcolor[rgb]{0.00,0.50,0.00}{##1}}}
\@namedef{PY@tok@nd}{\def\PY@tc##1{\textcolor[rgb]{0.67,0.13,1.00}{##1}}}
\@namedef{PY@tok@s}{\def\PY@tc##1{\textcolor[rgb]{0.73,0.13,0.13}{##1}}}
\@namedef{PY@tok@sd}{\let\PY@it=\textit\def\PY@tc##1{\textcolor[rgb]{0.73,0.13,0.13}{##1}}}
\@namedef{PY@tok@si}{\let\PY@bf=\textbf\def\PY@tc##1{\textcolor[rgb]{0.64,0.35,0.47}{##1}}}
\@namedef{PY@tok@se}{\let\PY@bf=\textbf\def\PY@tc##1{\textcolor[rgb]{0.67,0.36,0.12}{##1}}}
\@namedef{PY@tok@sr}{\def\PY@tc##1{\textcolor[rgb]{0.64,0.35,0.47}{##1}}}
\@namedef{PY@tok@ss}{\def\PY@tc##1{\textcolor[rgb]{0.10,0.09,0.49}{##1}}}
\@namedef{PY@tok@sx}{\def\PY@tc##1{\textcolor[rgb]{0.00,0.50,0.00}{##1}}}
\@namedef{PY@tok@m}{\def\PY@tc##1{\textcolor[rgb]{0.40,0.40,0.40}{##1}}}
\@namedef{PY@tok@gh}{\let\PY@bf=\textbf\def\PY@tc##1{\textcolor[rgb]{0.00,0.00,0.50}{##1}}}
\@namedef{PY@tok@gu}{\let\PY@bf=\textbf\def\PY@tc##1{\textcolor[rgb]{0.50,0.00,0.50}{##1}}}
\@namedef{PY@tok@gd}{\def\PY@tc##1{\textcolor[rgb]{0.63,0.00,0.00}{##1}}}
\@namedef{PY@tok@gi}{\def\PY@tc##1{\textcolor[rgb]{0.00,0.52,0.00}{##1}}}
\@namedef{PY@tok@gr}{\def\PY@tc##1{\textcolor[rgb]{0.89,0.00,0.00}{##1}}}
\@namedef{PY@tok@ge}{\let\PY@it=\textit}
\@namedef{PY@tok@gs}{\let\PY@bf=\textbf}
\@namedef{PY@tok@gp}{\let\PY@bf=\textbf\def\PY@tc##1{\textcolor[rgb]{0.00,0.00,0.50}{##1}}}
\@namedef{PY@tok@go}{\def\PY@tc##1{\textcolor[rgb]{0.44,0.44,0.44}{##1}}}
\@namedef{PY@tok@gt}{\def\PY@tc##1{\textcolor[rgb]{0.00,0.27,0.87}{##1}}}
\@namedef{PY@tok@err}{\def\PY@bc##1{{\setlength{\fboxsep}{\string -\fboxrule}\fcolorbox[rgb]{1.00,0.00,0.00}{1,1,1}{\strut ##1}}}}
\@namedef{PY@tok@kc}{\let\PY@bf=\textbf\def\PY@tc##1{\textcolor[rgb]{0.00,0.50,0.00}{##1}}}
\@namedef{PY@tok@kd}{\let\PY@bf=\textbf\def\PY@tc##1{\textcolor[rgb]{0.00,0.50,0.00}{##1}}}
\@namedef{PY@tok@kn}{\let\PY@bf=\textbf\def\PY@tc##1{\textcolor[rgb]{0.00,0.50,0.00}{##1}}}
\@namedef{PY@tok@kr}{\let\PY@bf=\textbf\def\PY@tc##1{\textcolor[rgb]{0.00,0.50,0.00}{##1}}}
\@namedef{PY@tok@bp}{\def\PY@tc##1{\textcolor[rgb]{0.00,0.50,0.00}{##1}}}
\@namedef{PY@tok@fm}{\def\PY@tc##1{\textcolor[rgb]{0.00,0.00,1.00}{##1}}}
\@namedef{PY@tok@vc}{\def\PY@tc##1{\textcolor[rgb]{0.10,0.09,0.49}{##1}}}
\@namedef{PY@tok@vg}{\def\PY@tc##1{\textcolor[rgb]{0.10,0.09,0.49}{##1}}}
\@namedef{PY@tok@vi}{\def\PY@tc##1{\textcolor[rgb]{0.10,0.09,0.49}{##1}}}
\@namedef{PY@tok@vm}{\def\PY@tc##1{\textcolor[rgb]{0.10,0.09,0.49}{##1}}}
\@namedef{PY@tok@sa}{\def\PY@tc##1{\textcolor[rgb]{0.73,0.13,0.13}{##1}}}
\@namedef{PY@tok@sb}{\def\PY@tc##1{\textcolor[rgb]{0.73,0.13,0.13}{##1}}}
\@namedef{PY@tok@sc}{\def\PY@tc##1{\textcolor[rgb]{0.73,0.13,0.13}{##1}}}
\@namedef{PY@tok@dl}{\def\PY@tc##1{\textcolor[rgb]{0.73,0.13,0.13}{##1}}}
\@namedef{PY@tok@s2}{\def\PY@tc##1{\textcolor[rgb]{0.73,0.13,0.13}{##1}}}
\@namedef{PY@tok@sh}{\def\PY@tc##1{\textcolor[rgb]{0.73,0.13,0.13}{##1}}}
\@namedef{PY@tok@s1}{\def\PY@tc##1{\textcolor[rgb]{0.73,0.13,0.13}{##1}}}
\@namedef{PY@tok@mb}{\def\PY@tc##1{\textcolor[rgb]{0.40,0.40,0.40}{##1}}}
\@namedef{PY@tok@mf}{\def\PY@tc##1{\textcolor[rgb]{0.40,0.40,0.40}{##1}}}
\@namedef{PY@tok@mh}{\def\PY@tc##1{\textcolor[rgb]{0.40,0.40,0.40}{##1}}}
\@namedef{PY@tok@mi}{\def\PY@tc##1{\textcolor[rgb]{0.40,0.40,0.40}{##1}}}
\@namedef{PY@tok@il}{\def\PY@tc##1{\textcolor[rgb]{0.40,0.40,0.40}{##1}}}
\@namedef{PY@tok@mo}{\def\PY@tc##1{\textcolor[rgb]{0.40,0.40,0.40}{##1}}}
\@namedef{PY@tok@ch}{\let\PY@it=\textit\def\PY@tc##1{\textcolor[rgb]{0.24,0.48,0.48}{##1}}}
\@namedef{PY@tok@cm}{\let\PY@it=\textit\def\PY@tc##1{\textcolor[rgb]{0.24,0.48,0.48}{##1}}}
\@namedef{PY@tok@cpf}{\let\PY@it=\textit\def\PY@tc##1{\textcolor[rgb]{0.24,0.48,0.48}{##1}}}
\@namedef{PY@tok@c1}{\let\PY@it=\textit\def\PY@tc##1{\textcolor[rgb]{0.24,0.48,0.48}{##1}}}
\@namedef{PY@tok@cs}{\let\PY@it=\textit\def\PY@tc##1{\textcolor[rgb]{0.24,0.48,0.48}{##1}}}

\def\PYZbs{\char`\\}
\def\PYZus{\char`\_}
\def\PYZob{\char`\{}
\def\PYZcb{\char`\}}
\def\PYZca{\char`\^}
\def\PYZam{\char`\&}
\def\PYZlt{\char`\<}
\def\PYZgt{\char`\>}
\def\PYZsh{\char`\#}
\def\PYZpc{\char`\%}
\def\PYZdl{\char`\$}
\def\PYZhy{\char`\-}
\def\PYZsq{\char`\'}
\def\PYZdq{\char`\"}
\def\PYZti{\char`\~}
% for compatibility with earlier versions
\def\PYZat{@}
\def\PYZlb{[}
\def\PYZrb{]}
\makeatother


    % For linebreaks inside Verbatim environment from package fancyvrb.
    \makeatletter
        \newbox\Wrappedcontinuationbox
        \newbox\Wrappedvisiblespacebox
        \newcommand*\Wrappedvisiblespace {\textcolor{red}{\textvisiblespace}}
        \newcommand*\Wrappedcontinuationsymbol {\textcolor{red}{\llap{\tiny$\m@th\hookrightarrow$}}}
        \newcommand*\Wrappedcontinuationindent {3ex }
        \newcommand*\Wrappedafterbreak {\kern\Wrappedcontinuationindent\copy\Wrappedcontinuationbox}
        % Take advantage of the already applied Pygments mark-up to insert
        % potential linebreaks for TeX processing.
        %        {, <, #, %, $, ' and ": go to next line.
        %        _, }, ^, &, >, - and ~: stay at end of broken line.
        % Use of \textquotesingle for straight quote.
        \newcommand*\Wrappedbreaksatspecials {%
            \def\PYGZus{\discretionary{\char`\_}{\Wrappedafterbreak}{\char`\_}}%
            \def\PYGZob{\discretionary{}{\Wrappedafterbreak\char`\{}{\char`\{}}%
            \def\PYGZcb{\discretionary{\char`\}}{\Wrappedafterbreak}{\char`\}}}%
            \def\PYGZca{\discretionary{\char`\^}{\Wrappedafterbreak}{\char`\^}}%
            \def\PYGZam{\discretionary{\char`\&}{\Wrappedafterbreak}{\char`\&}}%
            \def\PYGZlt{\discretionary{}{\Wrappedafterbreak\char`\<}{\char`\<}}%
            \def\PYGZgt{\discretionary{\char`\>}{\Wrappedafterbreak}{\char`\>}}%
            \def\PYGZsh{\discretionary{}{\Wrappedafterbreak\char`\#}{\char`\#}}%
            \def\PYGZpc{\discretionary{}{\Wrappedafterbreak\char`\%}{\char`\%}}%
            \def\PYGZdl{\discretionary{}{\Wrappedafterbreak\char`\$}{\char`\$}}%
            \def\PYGZhy{\discretionary{\char`\-}{\Wrappedafterbreak}{\char`\-}}%
            \def\PYGZsq{\discretionary{}{\Wrappedafterbreak\textquotesingle}{\textquotesingle}}%
            \def\PYGZdq{\discretionary{}{\Wrappedafterbreak\char`\"}{\char`\"}}%
            \def\PYGZti{\discretionary{\char`\~}{\Wrappedafterbreak}{\char`\~}}%
        }
        % Some characters . , ; ? ! / are not pygmentized.
        % This macro makes them "active" and they will insert potential linebreaks
        \newcommand*\Wrappedbreaksatpunct {%
            \lccode`\~`\.\lowercase{\def~}{\discretionary{\hbox{\char`\.}}{\Wrappedafterbreak}{\hbox{\char`\.}}}%
            \lccode`\~`\,\lowercase{\def~}{\discretionary{\hbox{\char`\,}}{\Wrappedafterbreak}{\hbox{\char`\,}}}%
            \lccode`\~`\;\lowercase{\def~}{\discretionary{\hbox{\char`\;}}{\Wrappedafterbreak}{\hbox{\char`\;}}}%
            \lccode`\~`\:\lowercase{\def~}{\discretionary{\hbox{\char`\:}}{\Wrappedafterbreak}{\hbox{\char`\:}}}%
            \lccode`\~`\?\lowercase{\def~}{\discretionary{\hbox{\char`\?}}{\Wrappedafterbreak}{\hbox{\char`\?}}}%
            \lccode`\~`\!\lowercase{\def~}{\discretionary{\hbox{\char`\!}}{\Wrappedafterbreak}{\hbox{\char`\!}}}%
            \lccode`\~`\/\lowercase{\def~}{\discretionary{\hbox{\char`\/}}{\Wrappedafterbreak}{\hbox{\char`\/}}}%
            \catcode`\.\active
            \catcode`\,\active
            \catcode`\;\active
            \catcode`\:\active
            \catcode`\?\active
            \catcode`\!\active
            \catcode`\/\active
            \lccode`\~`\~
        }
    \makeatother

    \let\OriginalVerbatim=\Verbatim
    \makeatletter
    \renewcommand{\Verbatim}[1][1]{%
        %\parskip\z@skip
        \sbox\Wrappedcontinuationbox {\Wrappedcontinuationsymbol}%
        \sbox\Wrappedvisiblespacebox {\FV@SetupFont\Wrappedvisiblespace}%
        \def\FancyVerbFormatLine ##1{\hsize\linewidth
            \vtop{\raggedright\hyphenpenalty\z@\exhyphenpenalty\z@
                \doublehyphendemerits\z@\finalhyphendemerits\z@
                \strut ##1\strut}%
        }%
        % If the linebreak is at a space, the latter will be displayed as visible
        % space at end of first line, and a continuation symbol starts next line.
        % Stretch/shrink are however usually zero for typewriter font.
        \def\FV@Space {%
            \nobreak\hskip\z@ plus\fontdimen3\font minus\fontdimen4\font
            \discretionary{\copy\Wrappedvisiblespacebox}{\Wrappedafterbreak}
            {\kern\fontdimen2\font}%
        }%

        % Allow breaks at special characters using \PYG... macros.
        \Wrappedbreaksatspecials
        % Breaks at punctuation characters . , ; ? ! and / need catcode=\active
        \OriginalVerbatim[#1,codes*=\Wrappedbreaksatpunct]%
    }
    \makeatother

    % Exact colors from NB
    \definecolor{incolor}{HTML}{303F9F}
    \definecolor{outcolor}{HTML}{D84315}
    \definecolor{cellborder}{HTML}{CFCFCF}
    \definecolor{cellbackground}{HTML}{F7F7F7}

    % prompt
    \makeatletter
    \newcommand{\boxspacing}{\kern\kvtcb@left@rule\kern\kvtcb@boxsep}
    \makeatother
    \newcommand{\prompt}[4]{
        {\ttfamily\llap{{\color{#2}[#3]:\hspace{3pt}#4}}\vspace{-\baselineskip}}
    }
    

    
    % Prevent overflowing lines due to hard-to-break entities
    \sloppy
    % Setup hyperref package
    \hypersetup{
      breaklinks=true,  % so long urls are correctly broken across lines
      colorlinks=true,
      urlcolor=urlcolor,
      linkcolor=linkcolor,
      citecolor=citecolor,
      }
    % Slightly bigger margins than the latex defaults
    
    \geometry{verbose,tmargin=1in,bmargin=1in,lmargin=1in,rmargin=1in}
    
    

\begin{document}
    
    \maketitle
    
    


\textbf{Тема:} Закони розподілу та числові характеристики випадкових
величин

\textbf{Мета:} набути практичних навичок у розв'язанні задач на
знаходження законів розподілу та числових характеристик дискретних та
неперервних випадкових величин, зокрема нормального закону, та
розв'язання типових задач по цій темі.

\section{Хід
роботи}\label{ux445ux456ux434-ux440ux43eux431ux43eux442ux438}

    \subsection{\texorpdfstring{\textbf{Задача
1}}{Задача 1}}\label{ux437ux430ux434ux430ux447ux430-1}

\textbf{Задача:} У мішень виконується 4 незалежних постріли з
ймовірністю влучення при кожному пострілі \(p = 0.8\). Необхідно: 1)
знайти закон розподілу ДВВ \(X\), що дорівнює кількості влучень у
мішень; 2) виразити функцію розподілу та функцію щільності розподілу ДВВ
за допомогою функції Хевісайда та \(\delta\)-функції Дірака; 3)
побудувати графіки функцій розподілу та щільності розподілу; 4) знайти
ймовірності подій \(1 \le X \le 3\) та \(X > 3\); 5) побудувати
багатокутник розподілу; 6) знайти математичне сподівання, дисперсію,
середнє квадратичне відхилення, теоретичні початкові та центральні
моменти 3 та 4 порядку; 7) знайти асиметрію та ексцес.

\textbf{Розв'язок:}

Маємо справу з \textbf{біноміальним законом розподілу}
\(X \sim B(n, p)\) з параметрами \(n=4\) (кількість пострілів) та
\(p=0.8\) (ймовірність влучення). Ймовірність промаху
\(q = 1 - p = 0.2\). Можливі значення \(X\) (кількість влучень):
\(\{0, 1, 2, 3, 4\}\). Ймовірності обчислюються за формулою Бернуллі:
\(P(X=k) = C_n^k p^k q^{n-k}\).

\textbf{1) Закон розподілу ДВВ \(X\)}

\begin{itemize}
\tightlist
\item
  \(P(X=0) = C_4^0 \cdot (0.8)^0 \cdot (0.2)^4 = 1 \cdot 1 \cdot 0.0016 = 0.0016\)
\item
  \(P(X=1) = C_4^1 \cdot (0.8)^1 \cdot (0.2)^3 = 4 \cdot 0.8 \cdot 0.008 = 0.0256\)
\item
  \(P(X=2) = C_4^2 \cdot (0.8)^2 \cdot (0.2)^2 = 6 \cdot 0.64 \cdot 0.04 = 0.1536\)
\item
  \(P(X=3) = C_4^3 \cdot (0.8)^3 \cdot (0.2)^1 = 4 \cdot 0.512 \cdot 0.2 = 0.4096\)
\item
  \(P(X=4) = C_4^4 \cdot (0.8)^4 \cdot (0.2)^0 = 1 \cdot 0.4096 \cdot 1 = 0.4096\)
\end{itemize}

Таблиця закону розподілу: 
\begin{center} 
    \begin{tabular}{|l|c|c|c|c|c|}
    \hline
    $k$ & 0 & 1 & 2 & 3 & 4 \\
    \hline
    $P(X=k)$ & 0.0016 & 0.0256 & 0.1536 & 0.4096 & 0.4096 \\
    \hline
    \end{tabular}
\end{center}



\emph{(Перевірка: \(0.0016 + 0.0256 + 0.1536 + 0.4096 + 0.4096 = 1.0\))}

\textbf{2) Функції розподілу та щільності}

\begin{itemize}
\tightlist
\item
  Функція щільності через \(\delta\)-функцію Дірака:
  \(f(x) = 0.0016\delta(x-0) + 0.0256\delta(x-1) + 0.1536\delta(x-2) + 0.4096\delta(x-3) + 0.4096\delta(x-4)\)
\item
  Функція розподілу через функцію Хевісайда \(H(x)\):
  \(F(x) = 0.0016H(x-0) + 0.0256H(x-1) + 0.1536H(x-2) + 0.4096H(x-3) + 0.4096H(x-4)\)
\end{itemize}

\textbf{3) Графіки функцій}

\begin{itemize}
\tightlist
\item
  \textbf{Графік функції щільності:} на осі Ox відкладаємо точки 0, 1,
  2, 3, 4. З цих точок проводимо вертикальні лінії (стовпчики) висотою,
  що дорівнює відповідній ймовірності (0.0016, 0.0256, 0.1536, 0.4096,
  0.4096).
\end{itemize}

\begin{figure}
\centering
\pandocbounded{\includegraphics[width=0.6\textwidth,alt={Графік функції щільності}]{Photo/5_1_1.png}}
\caption{Графік функції щільності}
\end{figure}

\begin{itemize}
\tightlist
\item
  \textbf{Графік функції розподілу:} це східчаста функція.

  \begin{itemize}
  \tightlist
  \item
    \(F(x) = 0\) при \(x \le 0\)
  \item
    \(F(x) = 0.0016\) при \(0 < x \le 1\)
  \item
    \(F(x) = 0.0016 + 0.0256 = 0.0272\) при \(1 < x \le 2\)
  \item
    \(F(x) = 0.0272 + 0.1536 = 0.1808\) при \(2 < x \le 3\)
  \item
    \(F(x) = 0.1808 + 0.4096 = 0.5904\) при \(3 < x \le 4\)
  \item
    \(F(x) = 0.5904 + 0.4096 = 1.0\) при \(x > 4\)
  \end{itemize}
\end{itemize}

\begin{figure}
\centering
\pandocbounded{\includegraphics[width=0.6\textwidth,alt={Графік функції розподілу}]{Photo/5_1_2.png}}
\caption{Графік функції розподілу}
\end{figure}

\textbf{4) Ймовірності подій}

\begin{itemize}
\tightlist
\item
  \(P(1 \le X \le 3) = P(X=1) + P(X=2) + P(X=3) = 0.0256 + 0.1536 + 0.4096 = 0.5888\)
\item
  \(P(X > 3) = P(X=4) = 0.4096\)
\end{itemize}

\textbf{5) Багатокутник розподілу}

На координатній площині відмічаємо точки з координатами \((k, P(X=k))\):
(0, 0.0016), (1, 0.0256), (2, 0.1536), (3, 0.4096), (4, 0.4096).
Послідовно з'єднуємо ці точки відрізками.

\begin{figure}
\centering
\pandocbounded{\includegraphics[width=0.6\textwidth,alt={Багатокутник розподілу}]{Photo/5_1_3.png}}
\caption{Багатокутник розподілу}
\end{figure}

\textbf{6) Числові характеристики}

\begin{itemize}
\tightlist
\item
  \textbf{Математичне сподівання:}
  \(M(X) = n \cdot p = 4 \cdot 0.8 = 3.2\)
\item
  \textbf{Дисперсія:}
  \(D(X) = n \cdot p \cdot q = 4 \cdot 0.8 \cdot 0.2 = 0.64\)
\item
  \textbf{Середнє квадратичне відхилення:}
  \(\sigma(X) = \sqrt{D(X)} = \sqrt{0.64} = 0.8\)
\item
  \textbf{Центральний момент 3-го порядку:}
  \(\mu_3 = n p q (q-p) = 4 \cdot 0.8 \cdot 0.2 \cdot (0.2 - 0.8) = 0.64 \cdot (-0.6) = -0.384\)
\item
  \textbf{Центральний момент 4-го порядку:}
  \(\mu_4 = n p q (1 + 3(n-2)pq) = 0.64 \cdot (1 + 3(4-2) \cdot 0.8 \cdot 0.2) = 0.64 \cdot (1 + 3 \cdot 2 \cdot 0.16) = 0.64 \cdot (1 + 0.96) = 0.64 \cdot 1.96 = 1.2544\)
\item
  \textbf{Початковий момент 3-го порядку:}
  \(\alpha_3 = M((X-M(X)+M(X))^3) = \mu_3 + 3\mu_2 M(X) + (M(X))^3 = -0.384 + 3 \cdot 0.64 \cdot 3.2 + (3.2)^3 = -0.384 + 6.144 + 32.768 = 38.528\)
\item
  \textbf{Початковий момент 4-го порядку:}
  \(\alpha_4 = \mu_4 + 4\mu_3 M(X) + 6\mu_2 (M(X))^2 + (M(X))^4 = 1.2544 + 4(-0.384)(3.2) + 6(0.64)(3.2)^2 + (3.2)^4 = 1.2544 - 4.9152 + 39.3216 + 104.8576 = 140.5184\)
\end{itemize}

\textbf{7) Асиметрія та ексцес}

\begin{itemize}
\tightlist
\item
  \textbf{Асиметрія:}
  \(As = \frac{\mu_3}{\sigma^3} = \frac{-0.384}{(0.8)^3} = \frac{-0.384}{0.512} = -0.75\)
\item
  \textbf{Ексцес:}
  \(Ex = \frac{\mu_4}{\sigma^4} - 3 = \frac{1.2544}{(0.8)^4} - 3 = \frac{1.2544}{0.4096} - 3 = 3.0625 - 3 = 0.0625\)
\end{itemize}

    \subsection{\texorpdfstring{\textbf{Задача
2}}{Задача 2}}\label{ux437ux430ux434ux430ux447ux430-2}

\textbf{Задача:} З ймовірністю влучення при одному пострілі \(p = 0.7\)
стрілок стріляє у мішень до першого влучення, але може виконати не
більше 4 пострілів. ДВВ \(X\) -- кількість промахів. Необхідно: 1)
знайти закон розподілу ДВВ; 2) виразити функцію розподілу та функцію
щільності розподілу ДВВ за допомогою функції Хевісайда та
\(\delta\)-функції Дірака; 3) побудувати графіки функцій розподілу та
щільності розподілу; 4) знайти асиметрію та ексцес.

\textbf{Розв'язок:}

\(X\) - кількість промахів. Ймовірність влучення \(p=0.7\), ймовірність
промаху \(q=1-p=0.3\). Можливі значення \(X\): \(\{0, 1, 2, 3\}\).

\textbf{1) Закон розподілу ДВВ}

\begin{itemize}
\tightlist
\item
  \(X=0\) (0 промахів): Перший постріл - влучення. \(P(X=0) = p = 0.7\)
\item
  \(X=1\) (1 промах): Перший - промах, другий - влучення.
  \(P(X=1) = q \cdot p = 0.3 \cdot 0.7 = 0.21\)
\item
  \(X=2\) (2 промахи): Перші два - промахи, третій - влучення.
  \(P(X=2) = q^2 \cdot p = (0.3)^2 \cdot 0.7 = 0.09 \cdot 0.7 = 0.063\)
\item
  \(X=3\) (3 промахи): Ця подія відбувається, якщо було 3 промахи і 4-й
  постріл - влучення, АБО якщо всі 4 постріли - промахи (бо більше
  стріляти не можна).
  \(P(X=3) = q^3 \cdot p + q^4 = q^3(p+q) = q^3 = (0.3)^3 = 0.027\)
\end{itemize}

Таблиця закону розподілу: 
\begin{center}
    \begin{tabular}{|l|c|c|c|c|}
    \hline
    $k$ (промахи) & 0 & 1 & 2 & 3 \\
    \hline
    $P(X=k)$ & 0.7 & 0.21 & 0.063 & 0.027 \\
    \hline
    \end{tabular}
\end{center}


\emph{(Перевірка: \(0.7 + 0.21 + 0.063 + 0.027 = 1.0\))}

\textbf{2) Функції розподілу та щільності}

\begin{itemize}
\tightlist
\item
  \(f(x) = 0.7\delta(x-0) + 0.21\delta(x-1) + 0.063\delta(x-2) + 0.027\delta(x-3)\)
\item
  \(F(x) = 0.7H(x-0) + 0.21H(x-1) + 0.063H(x-2) + 0.027H(x-3)\)
\end{itemize}

\textbf{3) Графіки функцій}

\begin{itemize}
\tightlist
\item
  \textbf{Графік функції щільності:} Стовпчики в точках 0, 1, 2, 3 з
  висотами 0.7, 0.21, 0.063, 0.027.
\end{itemize}

\begin{figure}
\centering
\pandocbounded{\includegraphics[width=0.6\textwidth,alt={Графік функції щільності}]{Photo/5_2_1.png}}
\caption{Графік функції щільності}
\end{figure}

\begin{itemize}
\tightlist
\item
  \textbf{Графік функції розподілу:} Східчаста функція, що зростає в
  точках 0, 1, 2, 3 на величину відповідних ймовірностей.
\end{itemize}

\begin{figure}
\centering
\pandocbounded{\includegraphics[width=0.6\textwidth,alt={Графік функції розподілу}]{Photo/5_2_2.png}}
\caption{Графік функції розподілу}
\end{figure}

\textbf{4) Асиметрія та ексцес}

Спочатку знайдемо моменти за визначенням: *
\(M(X) = \sum k \cdot P(X=k) = 0(0.7) + 1(0.21) + 2(0.063) + 3(0.027) = 0 + 0.21 + 0.126 + 0.081 = 0.417\)
*
\(M(X^2) = \sum k^2 \cdot P(X=k) = 0^2(0.7) + 1^2(0.21) + 2^2(0.063) + 3^2(0.027) = 0 + 0.21 + 4(0.063) + 9(0.027) = 0.21 + 0.252 + 0.243 = 0.705\)
*
\(D(X) = M(X^2) - [M(X)]^2 = 0.705 - (0.417)^2 = 0.705 - 0.173889 = 0.531111\)
* \(\sigma(X) = \sqrt{D(X)} \approx \sqrt{0.531111} \approx 0.7288\)

Для асиметрії та ексцесу потрібні \(\mu_3\) та \(\mu_4\): *
\(\mu_3 = \sum (k-M(X))^3 P(k) = (0-0.417)^3(0.7) + (1-0.417)^3(0.21) + (2-0.417)^3(0.063) + (3-0.417)^3(0.027) \approx 0.708\)
*
\(\mu_4 = \sum (k-M(X))^4 P(k) = (0-0.417)^4(0.7) + (1-0.417)^4(0.21) + (2-0.417)^4(0.063) + (3-0.417)^4(0.027) \approx 1.488\)

\begin{itemize}
\tightlist
\item
  \textbf{Асиметрія:}
  \(As = \frac{\mu_3}{\sigma^3} \approx \frac{0.708}{(0.7288)^3} \approx 1.828\)
\item
  \textbf{Ексцес:}
  \(Ex = \frac{\mu_4}{\sigma^4} - 3 \approx \frac{1.488}{(0.7288)^4} - 3 \approx 5.28 - 3 = 2.28\)
\end{itemize}

    \subsection{\texorpdfstring{\textbf{Задача
3}}{Задача 3}}\label{ux437ux430ux434ux430ux447ux430-3}

\textbf{Задача:} Двічі кинута гральна кістка. ДВВ \(X\) -- різниця між
кількістю очок при першому киданні та кількістю очок при другому
киданні. Необхідно: 1) знайти закон розподілу ДВВ; 2) побудувати графік
функції щільності розподілу ДВВ; 3) знайти ймовірність події
\(|X| < 2\).

\textbf{Розв'язок:}

Загальна кількість рівноможливих наслідків: \(N = 6 \times 6 = 36\).
\(X = K_1 - K_2\). Можливі значення \(X\) від \(1-6=-5\) до \(6-1=5\).

\textbf{1) Закон розподілу ДВВ}

Знайдемо кількість сприятливих наслідків (пар \((K_1, K_2)\)) для
кожного значення \(X\): * \(X=-5\): (1,6) - 1 наслідок * \(X=-4\):
(1,5), (2,6) - 2 наслідки * \(X=-3\): (1,4), (2,5), (3,6) - 3 наслідки *
\(X=-2\): (1,3), (2,4), (3,5), (4,6) - 4 наслідки * \(X=-1\): (1,2),
(2,3), (3,4), (4,5), (5,6) - 5 наслідків * \(X=0\): (1,1), (2,2), (3,3),
(4,4), (5,5), (6,6) - 6 наслідків * \(X=1\): (2,1), (3,2), (4,3), (5,4),
(6,5) - 5 наслідків * \(X=2\): (3,1), (4,2), (5,3), (6,4) - 4 наслідки *
\(X=3\): (4,1), (5,2), (6,3) - 3 наслідки * \(X=4\): (5,1), (6,2) - 2
наслідки * \(X=5\): (6,1) - 1 наслідок

Таблиця закону розподілу (ймовірність = кількість наслідків / 36):
\begin{center}
    \begin{tabular}{|l|c|c|c|c|c|c|c|c|c|c|c|}
    \hline
    $k$ & -5 & -4 & -3 & -4 & -1 & 0 & 1 & 2 & 3 & 4 & 5 \\
    \hline
    $P(X=k)$ & $\frac{1}{36}$ & $\frac{2}{36}$ & $\frac{3}{36}$ & $\frac{4}{36}$ & $\frac{5}{36}$ & $\frac{6}{36}$ & $\frac{5}{36}$ & $\frac{4}{36}$ & $\frac{3}{36}$ & $\frac{2}{36}$ & $\frac{1}{36}$ \\
    \hline
    \end{tabular}
\end{center}


\textbf{2) Графік функції щільності}

На осі Ox відкладаємо цілі числа від -5 до 5. З кожної точки проводимо
вертикальний стовпчик висотою, що дорівнює відповідній ймовірності.
Графік симетричний відносно осі Oy.

\begin{figure}
\centering
\pandocbounded{\includegraphics[width=0.6\textwidth,alt={Графік функції щільності}]{Photo/5_3_1.png}}
\caption{Графік функції щільності}
\end{figure}

\textbf{3) Ймовірність події \(|X| < 2\)}

Подія \(|X| < 2\) означає, що \(X\) може приймати значення -1, 0, 1.
\(P(|X| < 2) = P(X=-1) + P(X=0) + P(X=1) = \frac{5}{36} + \frac{6}{36} + \frac{5}{36} = \frac{16}{36} = \frac{4}{9} \approx 0.4444\)

    \subsection{\texorpdfstring{\textbf{Задача
4}}{Задача 4}}\label{ux437ux430ux434ux430ux447ux430-4}

\textbf{Задача:} В урні 7 кульок, з яких 4 білих, а інші -- чорні. З
цієї урни наугад беруть 3 кульки. ДВВ \(X\) -- кількість білих кульок.
Необхідно: 1) знайти закон розподілу ДВВ; 2) виразити функцію розподілу
та функцію щільності розподілу ДВВ за допомогою функції Хевісайда та
\(\delta\)-функції Дірака; 3) побудувати графіки функцій розподілу та
щільності розподілу; 4) знайти ймовірність події \(X \ge 1\); 5)
побудувати багатокутник розподілу; 6) знайти математичне сподівання,
дисперсію, середнє квадратичне відхилення, теоретичні початкові та
центральні моменти 3 та 4 порядку; 7) знайти асиметрію та ексцес.

\textbf{Розв'язок:}

Маємо справу з \textbf{гіпергеометричним законом розподілу}. \(N=7\)
(всього кульок), \(M=4\) (білих), \(N-M=3\) (чорних), \(n=3\) (беруть).
\(X\) - кількість білих серед взятих. Можливі значення \(X\):
\(\{0, 1, 2, 3\}\). Формула:
\(P(X=k) = \frac{C_M^k C_{N-M}^{n-k}}{C_N^n}\). Загальна кількість
способів взяти 3 з 7:
\(C_7^3 = \frac{7 \cdot 6 \cdot 5}{3 \cdot 2 \cdot 1} = 35\).

\textbf{1) Закон розподілу ДВВ}

\begin{itemize}
\tightlist
\item
  \(P(X=0) = \frac{C_4^0 C_3^3}{C_7^3} = \frac{1 \cdot 1}{35} = \frac{1}{35}\)
\item
  \(P(X=1) = \frac{C_4^1 C_3^2}{C_7^3} = \frac{4 \cdot 3}{35} = \frac{12}{35}\)
\item
  \(P(X=2) = \frac{C_4^2 C_3^1}{C_7^3} = \frac{6 \cdot 3}{35} = \frac{18}{35}\)
\item
  \(P(X=3) = \frac{C_4^3 C_3^0}{C_7^3} = \frac{4 \cdot 1}{35} = \frac{4}{35}\)
\end{itemize}

Таблиця: 
\begin{center}
    \begin{tabular}{|l|c|c|c|c|}
    \hline
    $k$ (білі кульки) & 0 & 1 & 2 & 3 \\
    \hline
    $P(X=k)$ & $\frac{1}{35}$ & $\frac{12}{35}$ & $\frac{18}{35}$ & $\frac{4}{35}$ \\
    \hline
    \end{tabular}
\end{center}


\textbf{2) Функції розподілу та щільності}

\begin{itemize}
\tightlist
\item
  \(f(x) = \frac{1}{35}\delta(x-0) + \frac{12}{35}\delta(x-1) + \frac{18}{35}\delta(x-2) + \frac{4}{35}\delta(x-3)\)
\item
  \(F(x) = \frac{1}{35}H(x-0) + \frac{12}{35}H(x-1) + \frac{18}{35}H(x-2) + \frac{4}{35}H(x-3)\)
\end{itemize}

\textbf{3) Графіки функцій} * \textbf{Графік функції щільності:}

\begin{figure}
\centering
\pandocbounded{\includegraphics[width=0.6\textwidth,alt={Графік функції щільності}]{Photo/5_4_1.png}}
\caption{Графік функції щільності}
\end{figure}

\begin{itemize}
\tightlist
\item
  \textbf{Графік функції розподілу:}
\end{itemize}

\begin{figure}
\centering
\pandocbounded{\includegraphics[width=0.6\textwidth,alt={Графік функції розподілу}]{Photo/5_4_2.png}}
\caption{Графік функції розподілу}
\end{figure}

\textbf{4) Ймовірність події \(X \ge 1\)}
\(P(X \ge 1) = 1 - P(X=0) = 1 - \frac{1}{35} = \frac{34}{35}\).

\textbf{5) Багатокутник розподілу}

\begin{figure}
\centering
\pandocbounded{\includegraphics[width=0.6\textwidth,alt={Багатокутник розподілу}]{Photo/5_4_3.png}}
\caption{Багатокутник розподілу}
\end{figure}

\textbf{6) Числові характеристики}

\begin{itemize}
\tightlist
\item
  \(M(X) = n \frac{M}{N} = 3 \cdot \frac{4}{7} = \frac{12}{7} \approx 1.714\)
\item
  \(D(X) = n \frac{M}{N} (1-\frac{M}{N}) \frac{N-n}{N-1} = 3 \cdot \frac{4}{7} \cdot \frac{3}{7} \cdot \frac{7-3}{7-1} = \frac{12}{7} \cdot \frac{3}{7} \cdot \frac{4}{6} = \frac{24}{49} \approx 0.490\)
\item
  \(\sigma(X) = \sqrt{D(X)} = \sqrt{\frac{24}{49}} = \frac{\sqrt{24}}{7} = \frac{2\sqrt{6}}{7} \approx 0.700\)
\end{itemize}

Розрахунок моментів за визначенням (як у задачі 2): *
\(M(X) = 0(\frac{1}{35}) + 1(\frac{12}{35}) + 2(\frac{18}{35}) + 3(\frac{4}{35}) = \frac{12+36+12}{35} = \frac{60}{35} = \frac{12}{7}\)
(збігається) * \(\mu_3 \approx -0.056\) * \(\mu_4 \approx 0.354\)

\textbf{7) Асиметрія та ексцес}

\begin{itemize}
\tightlist
\item
  \(As = \frac{\mu_3}{\sigma^3} \approx \frac{-0.056}{(0.700)^3} \approx -0.163\)
\item
  \(Ex = \frac{\mu_4}{\sigma^4} - 3 \approx \frac{0.354}{(0.700)^4} - 3 \approx 1.47 - 3 = -1.53\)
\end{itemize}

    \subsection{\texorpdfstring{\textbf{Задача
5}}{Задача 5}}\label{ux437ux430ux434ux430ux447ux430-5}

\textbf{Задача:} Завод відправив на базу 500 цілих деталей. Ймовірність
зіпсування кожної деталі в дорозі \(p = 0.002\). Знайти закон розподілу
ДВВ \(X\), що дорівнює кількості зіпсованих деталей, та знайти
ймовірності подій: пошкоджено менше 3 деталей; пошкоджено більше 2
деталей; пошкоджено хоча б одну деталь.

\textbf{Розв'язок:}

Маємо велику кількість випробувань \(n=500\) і малу ймовірність події
\(p=0.002\). Це випадок для апроксимації біноміального розподілу
\textbf{розподілом Пуассона}. Параметр розподілу
\(\lambda = n \cdot p = 500 \cdot 0.002 = 1\). Отже,
\(X \sim Po(\lambda=1)\). Формула Пуассона:
\(P(X=k) = \frac{\lambda^k e^{-\lambda}}{k!}\). Використовуємо
\(e^{-1} \approx 0.3679\).

\textbf{Закон розподілу ДВВ \(X\)}

\begin{itemize}
\tightlist
\item
  \(P(X=0) = \frac{1^0 e^{-1}}{0!} = e^{-1} \approx 0.3679\)
\item
  \(P(X=1) = \frac{1^1 e^{-1}}{1!} = e^{-1} \approx 0.3679\)
\item
  \(P(X=2) = \frac{1^2 e^{-1}}{2!} = \frac{e^{-1}}{2} \approx 0.1839\)
\item
  \(P(X=3) = \frac{1^3 e^{-1}}{3!} = \frac{e^{-1}}{6} \approx 0.0613\)
\item
  \(P(X=4) = \frac{1^4 e^{-1}}{4!} = \frac{e^{-1}}{24} \approx 0.0153\)
\item
  \ldots{} і так далі.
\end{itemize}

\textbf{Ймовірності подій}

\begin{itemize}
\tightlist
\item
  \textbf{Пошкоджено менше 3 деталей (\(X < 3\)):}
  \(P(X < 3) = P(X=0) + P(X=1) + P(X=2) \approx 0.3679 + 0.3679 + 0.1839 = 0.9197\)
\item
  \textbf{Пошкоджено більше 2 деталей (\(X > 2\)):}
  \(P(X > 2) = 1 - P(X \le 2) = 1 - (P(X=0) + P(X=1) + P(X=2)) \approx 1 - 0.9197 = 0.0803\)
\item
  \textbf{Пошкоджено хоча б одну деталь (\(X \ge 1\)):}
  \(P(X \ge 1) = 1 - P(X=0) \approx 1 - 0.3679 = 0.6321\)
\end{itemize}

    \section{Контрольні
питання}\label{ux43aux43eux43dux442ux440ux43eux43bux44cux43dux456-ux43fux438ux442ux430ux43dux43dux44f}

\begin{center}\rule{0.5\linewidth}{0.5pt}\end{center}

\textbf{1. Навести декілька прикладів дискретної випадкової величини.} *
Кількість влучень при 10 пострілах. * Число, що випало на гральній
кістці. * Кількість студентів, що сьогодні відсутні на парі. * Кількість
бракованих деталей у партії.

\begin{center}\rule{0.5\linewidth}{0.5pt}\end{center}

\textbf{2. Навести декілька прикладів неперервної випадкової величини.}
* Температура повітря в кімнаті. * Час, за який студент добирається до
університету. * Точна вага людини. * Напруга в електромережі.

\begin{center}\rule{0.5\linewidth}{0.5pt}\end{center}

\textbf{3. Чи для всіх розподілів існують математичне сподівання і
дисперсія?} Ні, не для всіх. Класичний приклад --- \textbf{розподіл
Коші}, для якого не існує ні математичного сподівання, ні дисперсії,
оскільки відповідні інтеграли розбігаються.

\begin{center}\rule{0.5\linewidth}{0.5pt}\end{center}

\textbf{4. Як виправдати використання математичного сподівання як
числової характеристики для розподілу, який не має скінченного
математичного сподівання?} Це неможливо. Якщо математичне сподівання не
існує (є нескінченним або невизначеним), його не можна використовувати
як числову характеристику. В таких випадках для опису ``центра''
розподілу використовують інші характеристики, наприклад,
\textbf{медіану} або \textbf{моду}.

\begin{center}\rule{0.5\linewidth}{0.5pt}\end{center}

\textbf{5. Яка форма закону розподілу є універсальною і може бути
застосовна як для ДВВ, так і для НВВ?} \textbf{Функція розподілу}
\(F(x) = P(X < x)\). Вона існує для будь-якої випадкової величини, як
дискретної (де вона є східчастою), так і неперервної (де вона є
неперервною функцією).

\begin{center}\rule{0.5\linewidth}{0.5pt}\end{center}

\textbf{6. Які альтернативні числові характеристики можна
використовувати для опису розподілу, якщо математичне сподівання не
відображає його повністю?} * \textbf{Медіана:} значення, яке ділить
розподіл навпіл (50\% значень менше, 50\% --- більше). * \textbf{Мода:}
значення, що зустрічається найчастіше. * \textbf{Квантилі, децилі,
перцентилі:} значення, що ділять розподіл на рівні частини (4, 10, 100
відповідно).

\begin{center}\rule{0.5\linewidth}{0.5pt}\end{center}

\textbf{7. У чому полягає ймовірнісний та статистичний сенс
математичного сподівання?} * \textbf{Ймовірнісний сенс:} Це
середньозважене всіх можливих значень випадкової величини, де вагами є
їхні ймовірності. * \textbf{Статистичний сенс:} Це середнє арифметичне
значення, до якого прагне результат при великій кількості повторних
експериментів (згідно із законом великих чисел).

\begin{center}\rule{0.5\linewidth}{0.5pt}\end{center}

\textbf{8. Чому важливо враховувати асиметрію та ексцес при аналізі
розподілу величин?} * \textbf{Асиметрія (As)} показує ``скошеність''
розподілу. Додатна асиметрія означає ``довгий хвіст'' праворуч, від'ємна
--- ліворуч. Це допомагає зрозуміти, чи переважають екстремально великі
чи малі значення. * \textbf{Ексцес (Ex)} показує ``гостроту'' піку
розподілу порівняно з нормальним. Додатний ексцес (``гострий пік'')
вказує на більшу ймовірність екстремальних значень (``важкі хвости'').

\begin{center}\rule{0.5\linewidth}{0.5pt}\end{center}

\textbf{9. Чому, якщо для певної ВВ не існує математичного сподівання,
то не існує дисперсія, асиметрія і ексцес? Відповідь обґрунтуйте.} Тому
що всі ці характеристики \textbf{обчислюються через математичне
сподівання}. * \textbf{Дисперсія:} \(D(X) = M[(X - M(X))^2]\). Якщо не
існує \(M(X)\), цю формулу неможливо обчислити. * \textbf{Асиметрія та
ексцес:} обчислюються через центральні моменти (\(\mu_3, \mu_4\)), які,
в свою чергу, є математичними сподіваннями степенів відхилення від
\(M(X)\). Наприклад, \(\mu_3 = M[(X - M(X))^3]\). Без \(M(X)\)
центральні моменти не визначені.

\begin{center}\rule{0.5\linewidth}{0.5pt}\end{center}

\textbf{10. Чому на практиці часто можна апріорі вважати розподіл ВВ
нормальним?} Через \textbf{Центральну граничну теорему (ЦГТ)}. Вона
стверджує, що сума великої кількості незалежних, однаково розподілених
випадкових величин (з будь-яким початковим розподілом) буде мати
розподіл, близький до нормального. Оскільки багато реальних процесів
(наприклад, похибки вимірювань) є результатом дії багатьох незалежних
факторів, їхній сумарний ефект часто підкоряється нормальному закону.


    % Add a bibliography block to the postdoc
    
    
    
\end{document}
