\documentclass[11pt]{article}

    \usepackage[breakable]{tcolorbox}
    \usepackage{parskip} % Stop auto-indenting (to mimic markdown behaviour)
    

    % Basic figure setup, for now with no caption control since it's done
    % automatically by Pandoc (which extracts ![](path) syntax from Markdown).
    \usepackage{graphicx}
    % Keep aspect ratio if custom image width or height is specified
    \setkeys{Gin}{keepaspectratio}
    % Maintain compatibility with old templates. Remove in nbconvert 6.0
    \let\Oldincludegraphics\includegraphics
    % Ensure that by default, figures have no caption (until we provide a
    % proper Figure object with a Caption API and a way to capture that
    % in the conversion process - todo).
    \usepackage{caption}
    \DeclareCaptionFormat{nocaption}{}
    \captionsetup{format=nocaption,aboveskip=0pt,belowskip=0pt}

    \usepackage{float}
    \floatplacement{figure}{H} % forces figures to be placed at the correct location
    \usepackage{xcolor} % Allow colors to be defined
    \usepackage{enumerate} % Needed for markdown enumerations to work
    \usepackage{geometry} % Used to adjust the document margins
    \usepackage{amsmath} % Equations
    \usepackage{amssymb} % Equations
    \usepackage{textcomp} % defines textquotesingle
    % Hack from http://tex.stackexchange.com/a/47451/13684:
    \AtBeginDocument{%
        \def\PYZsq{\textquotesingle}% Upright quotes in Pygmentized code
    }
    \usepackage{upquote} % Upright quotes for verbatim code
    \usepackage{eurosym} % defines \euro

    \usepackage{iftex}
    \ifPDFTeX
        \usepackage[T1]{fontenc}
        \IfFileExists{alphabeta.sty}{
              \usepackage{alphabeta}
          }{
              \usepackage[mathletters]{ucs}
              \usepackage[utf8x]{inputenc}
          }
    \else
        \usepackage{fontspec}
        \usepackage{unicode-math}
    \fi

    \usepackage{fancyvrb} % verbatim replacement that allows latex
    \usepackage{grffile} % extends the file name processing of package graphics
                         % to support a larger range
    \makeatletter % fix for old versions of grffile with XeLaTeX
    \@ifpackagelater{grffile}{2019/11/01}
    {
      % Do nothing on new versions
    }
    {
      \def\Gread@@xetex#1{%
        \IfFileExists{"\Gin@base".bb}%
        {\Gread@eps{\Gin@base.bb}}%
        {\Gread@@xetex@aux#1}%
      }
    }
    \makeatother
    \usepackage[Export]{adjustbox} % Used to constrain images to a maximum size
    \adjustboxset{max size={0.9\linewidth}{0.9\paperheight}}

    % The hyperref package gives us a pdf with properly built
    % internal navigation ('pdf bookmarks' for the table of contents,
    % internal cross-reference links, web links for URLs, etc.)
    \usepackage{hyperref}
    % The default LaTeX title has an obnoxious amount of whitespace. By default,
    % titling removes some of it. It also provides customization options.
    \usepackage{titling}
    \usepackage{longtable} % longtable support required by pandoc >1.10
    \usepackage{booktabs}  % table support for pandoc > 1.12.2
    \usepackage{array}     % table support for pandoc >= 2.11.3
    \usepackage{calc}      % table minipage width calculation for pandoc >= 2.11.1
    \usepackage[inline]{enumitem} % IRkernel/repr support (it uses the enumerate* environment)
    \usepackage[normalem]{ulem} % ulem is needed to support strikethroughs (\sout)
                                % normalem makes italics be italics, not underlines
    \usepackage{soul}      % strikethrough (\st) support for pandoc >= 3.0.0
    \usepackage{mathrsfs}
    

    
    % Colors for the hyperref package
    \definecolor{urlcolor}{rgb}{0,.145,.698}
    \definecolor{linkcolor}{rgb}{.71,0.21,0.01}
    \definecolor{citecolor}{rgb}{.12,.54,.11}

    % ANSI colors
    \definecolor{ansi-black}{HTML}{3E424D}
    \definecolor{ansi-black-intense}{HTML}{282C36}
    \definecolor{ansi-red}{HTML}{E75C58}
    \definecolor{ansi-red-intense}{HTML}{B22B31}
    \definecolor{ansi-green}{HTML}{00A250}
    \definecolor{ansi-green-intense}{HTML}{007427}
    \definecolor{ansi-yellow}{HTML}{DDB62B}
    \definecolor{ansi-yellow-intense}{HTML}{B27D12}
    \definecolor{ansi-blue}{HTML}{208FFB}
    \definecolor{ansi-blue-intense}{HTML}{0065CA}
    \definecolor{ansi-magenta}{HTML}{D160C4}
    \definecolor{ansi-magenta-intense}{HTML}{A03196}
    \definecolor{ansi-cyan}{HTML}{60C6C8}
    \definecolor{ansi-cyan-intense}{HTML}{258F8F}
    \definecolor{ansi-white}{HTML}{C5C1B4}
    \definecolor{ansi-white-intense}{HTML}{A1A6B2}
    \definecolor{ansi-default-inverse-fg}{HTML}{FFFFFF}
    \definecolor{ansi-default-inverse-bg}{HTML}{000000}

    % common color for the border for error outputs.
    \definecolor{outerrorbackground}{HTML}{FFDFDF}

    % commands and environments needed by pandoc snippets
    % extracted from the output of `pandoc -s`
    \providecommand{\tightlist}{%
      \setlength{\itemsep}{0pt}\setlength{\parskip}{0pt}}
    \DefineVerbatimEnvironment{Highlighting}{Verbatim}{commandchars=\\\{\}}
    % Add ',fontsize=\small' for more characters per line
    \newenvironment{Shaded}{}{}
    \newcommand{\KeywordTok}[1]{\textcolor[rgb]{0.00,0.44,0.13}{\textbf{{#1}}}}
    \newcommand{\DataTypeTok}[1]{\textcolor[rgb]{0.56,0.13,0.00}{{#1}}}
    \newcommand{\DecValTok}[1]{\textcolor[rgb]{0.25,0.63,0.44}{{#1}}}
    \newcommand{\BaseNTok}[1]{\textcolor[rgb]{0.25,0.63,0.44}{{#1}}}
    \newcommand{\FloatTok}[1]{\textcolor[rgb]{0.25,0.63,0.44}{{#1}}}
    \newcommand{\CharTok}[1]{\textcolor[rgb]{0.25,0.44,0.63}{{#1}}}
    \newcommand{\StringTok}[1]{\textcolor[rgb]{0.25,0.44,0.63}{{#1}}}
    \newcommand{\CommentTok}[1]{\textcolor[rgb]{0.38,0.63,0.69}{\textit{{#1}}}}
    \newcommand{\OtherTok}[1]{\textcolor[rgb]{0.00,0.44,0.13}{{#1}}}
    \newcommand{\AlertTok}[1]{\textcolor[rgb]{1.00,0.00,0.00}{\textbf{{#1}}}}
    \newcommand{\FunctionTok}[1]{\textcolor[rgb]{0.02,0.16,0.49}{{#1}}}
    \newcommand{\RegionMarkerTok}[1]{{#1}}
    \newcommand{\ErrorTok}[1]{\textcolor[rgb]{1.00,0.00,0.00}{\textbf{{#1}}}}
    \newcommand{\NormalTok}[1]{{#1}}

    % Additional commands for more recent versions of Pandoc
    \newcommand{\ConstantTok}[1]{\textcolor[rgb]{0.53,0.00,0.00}{{#1}}}
    \newcommand{\SpecialCharTok}[1]{\textcolor[rgb]{0.25,0.44,0.63}{{#1}}}
    \newcommand{\VerbatimStringTok}[1]{\textcolor[rgb]{0.25,0.44,0.63}{{#1}}}
    \newcommand{\SpecialStringTok}[1]{\textcolor[rgb]{0.73,0.40,0.53}{{#1}}}
    \newcommand{\ImportTok}[1]{{#1}}
    \newcommand{\DocumentationTok}[1]{\textcolor[rgb]{0.73,0.13,0.13}{\textit{{#1}}}}
    \newcommand{\AnnotationTok}[1]{\textcolor[rgb]{0.38,0.63,0.69}{\textbf{\textit{{#1}}}}}
    \newcommand{\CommentVarTok}[1]{\textcolor[rgb]{0.38,0.63,0.69}{\textbf{\textit{{#1}}}}}
    \newcommand{\VariableTok}[1]{\textcolor[rgb]{0.10,0.09,0.49}{{#1}}}
    \newcommand{\ControlFlowTok}[1]{\textcolor[rgb]{0.00,0.44,0.13}{\textbf{{#1}}}}
    \newcommand{\OperatorTok}[1]{\textcolor[rgb]{0.40,0.40,0.40}{{#1}}}
    \newcommand{\BuiltInTok}[1]{{#1}}
    \newcommand{\ExtensionTok}[1]{{#1}}
    \newcommand{\PreprocessorTok}[1]{\textcolor[rgb]{0.74,0.48,0.00}{{#1}}}
    \newcommand{\AttributeTok}[1]{\textcolor[rgb]{0.49,0.56,0.16}{{#1}}}
    \newcommand{\InformationTok}[1]{\textcolor[rgb]{0.38,0.63,0.69}{\textbf{\textit{{#1}}}}}
    \newcommand{\WarningTok}[1]{\textcolor[rgb]{0.38,0.63,0.69}{\textbf{\textit{{#1}}}}}
    \makeatletter
    \newsavebox\pandoc@box
    \newcommand*\pandocbounded[1]{%
      \sbox\pandoc@box{#1}%
      % scaling factors for width and height
      \Gscale@div\@tempa\textheight{\dimexpr\ht\pandoc@box+\dp\pandoc@box\relax}%
      \Gscale@div\@tempb\linewidth{\wd\pandoc@box}%
      % select the smaller of both
      \ifdim\@tempb\p@<\@tempa\p@
        \let\@tempa\@tempb
      \fi
      % scaling accordingly (\@tempa < 1)
      \ifdim\@tempa\p@<\p@
        \scalebox{\@tempa}{\usebox\pandoc@box}%
      % scaling not needed, use as it is
      \else
        \usebox{\pandoc@box}%
      \fi
    }
    \makeatother

    % Define a nice break command that doesn't care if a line doesn't already
    % exist.
    \def\br{\hspace*{\fill} \\* }
    % Math Jax compatibility definitions
    \def\gt{>}
    \def\lt{<}
    \let\Oldtex\TeX
    \let\Oldlatex\LaTeX
    \renewcommand{\TeX}{\textrm{\Oldtex}}
    \renewcommand{\LaTeX}{\textrm{\Oldlatex}}
    % Document parameters
    % Document title
    \title{Report\_Pz\_6}
    
    
    
    
    
    
    
% Pygments definitions
\makeatletter
\def\PY@reset{\let\PY@it=\relax \let\PY@bf=\relax%
    \let\PY@ul=\relax \let\PY@tc=\relax%
    \let\PY@bc=\relax \let\PY@ff=\relax}
\def\PY@tok#1{\csname PY@tok@#1\endcsname}
\def\PY@toks#1+{\ifx\relax#1\empty\else%
    \PY@tok{#1}\expandafter\PY@toks\fi}
\def\PY@do#1{\PY@bc{\PY@tc{\PY@ul{%
    \PY@it{\PY@bf{\PY@ff{#1}}}}}}}
\def\PY#1#2{\PY@reset\PY@toks#1+\relax+\PY@do{#2}}

\@namedef{PY@tok@w}{\def\PY@tc##1{\textcolor[rgb]{0.73,0.73,0.73}{##1}}}
\@namedef{PY@tok@c}{\let\PY@it=\textit\def\PY@tc##1{\textcolor[rgb]{0.24,0.48,0.48}{##1}}}
\@namedef{PY@tok@cp}{\def\PY@tc##1{\textcolor[rgb]{0.61,0.40,0.00}{##1}}}
\@namedef{PY@tok@k}{\let\PY@bf=\textbf\def\PY@tc##1{\textcolor[rgb]{0.00,0.50,0.00}{##1}}}
\@namedef{PY@tok@kp}{\def\PY@tc##1{\textcolor[rgb]{0.00,0.50,0.00}{##1}}}
\@namedef{PY@tok@kt}{\def\PY@tc##1{\textcolor[rgb]{0.69,0.00,0.25}{##1}}}
\@namedef{PY@tok@o}{\def\PY@tc##1{\textcolor[rgb]{0.40,0.40,0.40}{##1}}}
\@namedef{PY@tok@ow}{\let\PY@bf=\textbf\def\PY@tc##1{\textcolor[rgb]{0.67,0.13,1.00}{##1}}}
\@namedef{PY@tok@nb}{\def\PY@tc##1{\textcolor[rgb]{0.00,0.50,0.00}{##1}}}
\@namedef{PY@tok@nf}{\def\PY@tc##1{\textcolor[rgb]{0.00,0.00,1.00}{##1}}}
\@namedef{PY@tok@nc}{\let\PY@bf=\textbf\def\PY@tc##1{\textcolor[rgb]{0.00,0.00,1.00}{##1}}}
\@namedef{PY@tok@nn}{\let\PY@bf=\textbf\def\PY@tc##1{\textcolor[rgb]{0.00,0.00,1.00}{##1}}}
\@namedef{PY@tok@ne}{\let\PY@bf=\textbf\def\PY@tc##1{\textcolor[rgb]{0.80,0.25,0.22}{##1}}}
\@namedef{PY@tok@nv}{\def\PY@tc##1{\textcolor[rgb]{0.10,0.09,0.49}{##1}}}
\@namedef{PY@tok@no}{\def\PY@tc##1{\textcolor[rgb]{0.53,0.00,0.00}{##1}}}
\@namedef{PY@tok@nl}{\def\PY@tc##1{\textcolor[rgb]{0.46,0.46,0.00}{##1}}}
\@namedef{PY@tok@ni}{\let\PY@bf=\textbf\def\PY@tc##1{\textcolor[rgb]{0.44,0.44,0.44}{##1}}}
\@namedef{PY@tok@na}{\def\PY@tc##1{\textcolor[rgb]{0.41,0.47,0.13}{##1}}}
\@namedef{PY@tok@nt}{\let\PY@bf=\textbf\def\PY@tc##1{\textcolor[rgb]{0.00,0.50,0.00}{##1}}}
\@namedef{PY@tok@nd}{\def\PY@tc##1{\textcolor[rgb]{0.67,0.13,1.00}{##1}}}
\@namedef{PY@tok@s}{\def\PY@tc##1{\textcolor[rgb]{0.73,0.13,0.13}{##1}}}
\@namedef{PY@tok@sd}{\let\PY@it=\textit\def\PY@tc##1{\textcolor[rgb]{0.73,0.13,0.13}{##1}}}
\@namedef{PY@tok@si}{\let\PY@bf=\textbf\def\PY@tc##1{\textcolor[rgb]{0.64,0.35,0.47}{##1}}}
\@namedef{PY@tok@se}{\let\PY@bf=\textbf\def\PY@tc##1{\textcolor[rgb]{0.67,0.36,0.12}{##1}}}
\@namedef{PY@tok@sr}{\def\PY@tc##1{\textcolor[rgb]{0.64,0.35,0.47}{##1}}}
\@namedef{PY@tok@ss}{\def\PY@tc##1{\textcolor[rgb]{0.10,0.09,0.49}{##1}}}
\@namedef{PY@tok@sx}{\def\PY@tc##1{\textcolor[rgb]{0.00,0.50,0.00}{##1}}}
\@namedef{PY@tok@m}{\def\PY@tc##1{\textcolor[rgb]{0.40,0.40,0.40}{##1}}}
\@namedef{PY@tok@gh}{\let\PY@bf=\textbf\def\PY@tc##1{\textcolor[rgb]{0.00,0.00,0.50}{##1}}}
\@namedef{PY@tok@gu}{\let\PY@bf=\textbf\def\PY@tc##1{\textcolor[rgb]{0.50,0.00,0.50}{##1}}}
\@namedef{PY@tok@gd}{\def\PY@tc##1{\textcolor[rgb]{0.63,0.00,0.00}{##1}}}
\@namedef{PY@tok@gi}{\def\PY@tc##1{\textcolor[rgb]{0.00,0.52,0.00}{##1}}}
\@namedef{PY@tok@gr}{\def\PY@tc##1{\textcolor[rgb]{0.89,0.00,0.00}{##1}}}
\@namedef{PY@tok@ge}{\let\PY@it=\textit}
\@namedef{PY@tok@gs}{\let\PY@bf=\textbf}
\@namedef{PY@tok@gp}{\let\PY@bf=\textbf\def\PY@tc##1{\textcolor[rgb]{0.00,0.00,0.50}{##1}}}
\@namedef{PY@tok@go}{\def\PY@tc##1{\textcolor[rgb]{0.44,0.44,0.44}{##1}}}
\@namedef{PY@tok@gt}{\def\PY@tc##1{\textcolor[rgb]{0.00,0.27,0.87}{##1}}}
\@namedef{PY@tok@err}{\def\PY@bc##1{{\setlength{\fboxsep}{\string -\fboxrule}\fcolorbox[rgb]{1.00,0.00,0.00}{1,1,1}{\strut ##1}}}}
\@namedef{PY@tok@kc}{\let\PY@bf=\textbf\def\PY@tc##1{\textcolor[rgb]{0.00,0.50,0.00}{##1}}}
\@namedef{PY@tok@kd}{\let\PY@bf=\textbf\def\PY@tc##1{\textcolor[rgb]{0.00,0.50,0.00}{##1}}}
\@namedef{PY@tok@kn}{\let\PY@bf=\textbf\def\PY@tc##1{\textcolor[rgb]{0.00,0.50,0.00}{##1}}}
\@namedef{PY@tok@kr}{\let\PY@bf=\textbf\def\PY@tc##1{\textcolor[rgb]{0.00,0.50,0.00}{##1}}}
\@namedef{PY@tok@bp}{\def\PY@tc##1{\textcolor[rgb]{0.00,0.50,0.00}{##1}}}
\@namedef{PY@tok@fm}{\def\PY@tc##1{\textcolor[rgb]{0.00,0.00,1.00}{##1}}}
\@namedef{PY@tok@vc}{\def\PY@tc##1{\textcolor[rgb]{0.10,0.09,0.49}{##1}}}
\@namedef{PY@tok@vg}{\def\PY@tc##1{\textcolor[rgb]{0.10,0.09,0.49}{##1}}}
\@namedef{PY@tok@vi}{\def\PY@tc##1{\textcolor[rgb]{0.10,0.09,0.49}{##1}}}
\@namedef{PY@tok@vm}{\def\PY@tc##1{\textcolor[rgb]{0.10,0.09,0.49}{##1}}}
\@namedef{PY@tok@sa}{\def\PY@tc##1{\textcolor[rgb]{0.73,0.13,0.13}{##1}}}
\@namedef{PY@tok@sb}{\def\PY@tc##1{\textcolor[rgb]{0.73,0.13,0.13}{##1}}}
\@namedef{PY@tok@sc}{\def\PY@tc##1{\textcolor[rgb]{0.73,0.13,0.13}{##1}}}
\@namedef{PY@tok@dl}{\def\PY@tc##1{\textcolor[rgb]{0.73,0.13,0.13}{##1}}}
\@namedef{PY@tok@s2}{\def\PY@tc##1{\textcolor[rgb]{0.73,0.13,0.13}{##1}}}
\@namedef{PY@tok@sh}{\def\PY@tc##1{\textcolor[rgb]{0.73,0.13,0.13}{##1}}}
\@namedef{PY@tok@s1}{\def\PY@tc##1{\textcolor[rgb]{0.73,0.13,0.13}{##1}}}
\@namedef{PY@tok@mb}{\def\PY@tc##1{\textcolor[rgb]{0.40,0.40,0.40}{##1}}}
\@namedef{PY@tok@mf}{\def\PY@tc##1{\textcolor[rgb]{0.40,0.40,0.40}{##1}}}
\@namedef{PY@tok@mh}{\def\PY@tc##1{\textcolor[rgb]{0.40,0.40,0.40}{##1}}}
\@namedef{PY@tok@mi}{\def\PY@tc##1{\textcolor[rgb]{0.40,0.40,0.40}{##1}}}
\@namedef{PY@tok@il}{\def\PY@tc##1{\textcolor[rgb]{0.40,0.40,0.40}{##1}}}
\@namedef{PY@tok@mo}{\def\PY@tc##1{\textcolor[rgb]{0.40,0.40,0.40}{##1}}}
\@namedef{PY@tok@ch}{\let\PY@it=\textit\def\PY@tc##1{\textcolor[rgb]{0.24,0.48,0.48}{##1}}}
\@namedef{PY@tok@cm}{\let\PY@it=\textit\def\PY@tc##1{\textcolor[rgb]{0.24,0.48,0.48}{##1}}}
\@namedef{PY@tok@cpf}{\let\PY@it=\textit\def\PY@tc##1{\textcolor[rgb]{0.24,0.48,0.48}{##1}}}
\@namedef{PY@tok@c1}{\let\PY@it=\textit\def\PY@tc##1{\textcolor[rgb]{0.24,0.48,0.48}{##1}}}
\@namedef{PY@tok@cs}{\let\PY@it=\textit\def\PY@tc##1{\textcolor[rgb]{0.24,0.48,0.48}{##1}}}

\def\PYZbs{\char`\\}
\def\PYZus{\char`\_}
\def\PYZob{\char`\{}
\def\PYZcb{\char`\}}
\def\PYZca{\char`\^}
\def\PYZam{\char`\&}
\def\PYZlt{\char`\<}
\def\PYZgt{\char`\>}
\def\PYZsh{\char`\#}
\def\PYZpc{\char`\%}
\def\PYZdl{\char`\$}
\def\PYZhy{\char`\-}
\def\PYZsq{\char`\'}
\def\PYZdq{\char`\"}
\def\PYZti{\char`\~}
% for compatibility with earlier versions
\def\PYZat{@}
\def\PYZlb{[}
\def\PYZrb{]}
\makeatother


    % For linebreaks inside Verbatim environment from package fancyvrb.
    \makeatletter
        \newbox\Wrappedcontinuationbox
        \newbox\Wrappedvisiblespacebox
        \newcommand*\Wrappedvisiblespace {\textcolor{red}{\textvisiblespace}}
        \newcommand*\Wrappedcontinuationsymbol {\textcolor{red}{\llap{\tiny$\m@th\hookrightarrow$}}}
        \newcommand*\Wrappedcontinuationindent {3ex }
        \newcommand*\Wrappedafterbreak {\kern\Wrappedcontinuationindent\copy\Wrappedcontinuationbox}
        % Take advantage of the already applied Pygments mark-up to insert
        % potential linebreaks for TeX processing.
        %        {, <, #, %, $, ' and ": go to next line.
        %        _, }, ^, &, >, - and ~: stay at end of broken line.
        % Use of \textquotesingle for straight quote.
        \newcommand*\Wrappedbreaksatspecials {%
            \def\PYGZus{\discretionary{\char`\_}{\Wrappedafterbreak}{\char`\_}}%
            \def\PYGZob{\discretionary{}{\Wrappedafterbreak\char`\{}{\char`\{}}%
            \def\PYGZcb{\discretionary{\char`\}}{\Wrappedafterbreak}{\char`\}}}%
            \def\PYGZca{\discretionary{\char`\^}{\Wrappedafterbreak}{\char`\^}}%
            \def\PYGZam{\discretionary{\char`\&}{\Wrappedafterbreak}{\char`\&}}%
            \def\PYGZlt{\discretionary{}{\Wrappedafterbreak\char`\<}{\char`\<}}%
            \def\PYGZgt{\discretionary{\char`\>}{\Wrappedafterbreak}{\char`\>}}%
            \def\PYGZsh{\discretionary{}{\Wrappedafterbreak\char`\#}{\char`\#}}%
            \def\PYGZpc{\discretionary{}{\Wrappedafterbreak\char`\%}{\char`\%}}%
            \def\PYGZdl{\discretionary{}{\Wrappedafterbreak\char`\$}{\char`\$}}%
            \def\PYGZhy{\discretionary{\char`\-}{\Wrappedafterbreak}{\char`\-}}%
            \def\PYGZsq{\discretionary{}{\Wrappedafterbreak\textquotesingle}{\textquotesingle}}%
            \def\PYGZdq{\discretionary{}{\Wrappedafterbreak\char`\"}{\char`\"}}%
            \def\PYGZti{\discretionary{\char`\~}{\Wrappedafterbreak}{\char`\~}}%
        }
        % Some characters . , ; ? ! / are not pygmentized.
        % This macro makes them "active" and they will insert potential linebreaks
        \newcommand*\Wrappedbreaksatpunct {%
            \lccode`\~`\.\lowercase{\def~}{\discretionary{\hbox{\char`\.}}{\Wrappedafterbreak}{\hbox{\char`\.}}}%
            \lccode`\~`\,\lowercase{\def~}{\discretionary{\hbox{\char`\,}}{\Wrappedafterbreak}{\hbox{\char`\,}}}%
            \lccode`\~`\;\lowercase{\def~}{\discretionary{\hbox{\char`\;}}{\Wrappedafterbreak}{\hbox{\char`\;}}}%
            \lccode`\~`\:\lowercase{\def~}{\discretionary{\hbox{\char`\:}}{\Wrappedafterbreak}{\hbox{\char`\:}}}%
            \lccode`\~`\?\lowercase{\def~}{\discretionary{\hbox{\char`\?}}{\Wrappedafterbreak}{\hbox{\char`\?}}}%
            \lccode`\~`\!\lowercase{\def~}{\discretionary{\hbox{\char`\!}}{\Wrappedafterbreak}{\hbox{\char`\!}}}%
            \lccode`\~`\/\lowercase{\def~}{\discretionary{\hbox{\char`\/}}{\Wrappedafterbreak}{\hbox{\char`\/}}}%
            \catcode`\.\active
            \catcode`\,\active
            \catcode`\;\active
            \catcode`\:\active
            \catcode`\?\active
            \catcode`\!\active
            \catcode`\/\active
            \lccode`\~`\~
        }
    \makeatother

    \let\OriginalVerbatim=\Verbatim
    \makeatletter
    \renewcommand{\Verbatim}[1][1]{%
        %\parskip\z@skip
        \sbox\Wrappedcontinuationbox {\Wrappedcontinuationsymbol}%
        \sbox\Wrappedvisiblespacebox {\FV@SetupFont\Wrappedvisiblespace}%
        \def\FancyVerbFormatLine ##1{\hsize\linewidth
            \vtop{\raggedright\hyphenpenalty\z@\exhyphenpenalty\z@
                \doublehyphendemerits\z@\finalhyphendemerits\z@
                \strut ##1\strut}%
        }%
        % If the linebreak is at a space, the latter will be displayed as visible
        % space at end of first line, and a continuation symbol starts next line.
        % Stretch/shrink are however usually zero for typewriter font.
        \def\FV@Space {%
            \nobreak\hskip\z@ plus\fontdimen3\font minus\fontdimen4\font
            \discretionary{\copy\Wrappedvisiblespacebox}{\Wrappedafterbreak}
            {\kern\fontdimen2\font}%
        }%

        % Allow breaks at special characters using \PYG... macros.
        \Wrappedbreaksatspecials
        % Breaks at punctuation characters . , ; ? ! and / need catcode=\active
        \OriginalVerbatim[#1,codes*=\Wrappedbreaksatpunct]%
    }
    \makeatother

    % Exact colors from NB
    \definecolor{incolor}{HTML}{303F9F}
    \definecolor{outcolor}{HTML}{D84315}
    \definecolor{cellborder}{HTML}{CFCFCF}
    \definecolor{cellbackground}{HTML}{F7F7F7}

    % prompt
    \makeatletter
    \newcommand{\boxspacing}{\kern\kvtcb@left@rule\kern\kvtcb@boxsep}
    \makeatother
    \newcommand{\prompt}[4]{
        {\ttfamily\llap{{\color{#2}[#3]:\hspace{3pt}#4}}\vspace{-\baselineskip}}
    }
    

    
    % Prevent overflowing lines due to hard-to-break entities
    \sloppy
    % Setup hyperref package
    \hypersetup{
      breaklinks=true,  % so long urls are correctly broken across lines
      colorlinks=true,
      urlcolor=urlcolor,
      linkcolor=linkcolor,
      citecolor=citecolor,
      }
    % Slightly bigger margins than the latex defaults
    
    \geometry{verbose,tmargin=1in,bmargin=1in,lmargin=1in,rmargin=1in}
    
    

\begin{document}
    
    \maketitle
    
    

    
    \section{Практична робота
№6}\label{ux43fux440ux430ux43aux442ux438ux447ux43dux430-ux440ux43eux431ux43eux442ux430-6}

\textbf{Виконав} студентка групи КН-24-1 Процко П.Д.

\textbf{Тема:} Закони розподілу функцій випадкових величин. Композиція
законів розподілу. Розподіл екстремальних значень.

\textbf{Мета:} набути практичних навичок у розв'язанні задач з
обчислення функцій від випадкових величин, їх законів розподілу та
числових характеристик.

\section{Хід
роботи}\label{ux445ux456ux434-ux440ux43eux431ux43eux442ux438}

    \subsection{Задача 1}\label{ux437ux430ux434ux430ux447ux430-1}

\textbf{Задача:} \(X \sim U(a;b)\). Знайти закон розподілу ВВ
\(Z = \min(X)\).

\textbf{Формулювання:} \(X \sim U(a;b)\). Знайти закон розподілу ВВ
\(Z = \min(X_1, \dots, X_n)\), де \(X_i\) --- незалежна вибірка з
розподілу \(X\).

\textbf{Розв'язок:}

Функція розподілу для мінімуму з \(n\) незалежних однаково розподілених
випадкових величин знаходиться за формулою:
\[ F_Z(z) = 1 - (1 - F_X(z))^n \]

Для рівномірного розподілу \(U(a,b)\) функція розподілу має вигляд:
\[ F_X(z) = \begin{cases} 0, & z < a \\ \frac{z-a}{b-a}, & a \le z \le b \\ 1, & z > b \end{cases} \]

Підставляємо \(F_X(z)\) у формулу для \(F_Z(z)\) для інтервалу
\([a, b]\):
\[ F_Z(z) = 1 - \left(1 - \frac{z-a}{b-a}\right)^n = 1 - \left(\frac{b-a - (z-a)}{b-a}\right)^n = 1 - \left(\frac{b-z}{b-a}\right)^n \]

\textbf{Відповідь:} Функція розподілу для \(Z = \min(X_1, \dots, X_n)\):
\[ F_Z(z) = \begin{cases} 0, & z < a \\ 1 - \left(\frac{b-z}{b-a}\right)^n, & a \le z \le b \\ 1, & z > b \end{cases} \]

    \subsection{Задача 18}\label{ux437ux430ux434ux430ux447ux430-18}

\textbf{Задача:} Шляхом експериментальних досліджень встановлено, що
похибка, зумовлена шумом аналогових елементів аналого-цифрового
перетворювача (АЦП) комп'ютеризованої системи контролю (КСК), --
випадкова величина \(X\), що має нормальний закон розподілу з
параметрами \(\mu = 0, \sigma = 1\), а похибка квантування -- випадкова
величина \(Y\), що має рівномірний розподіл з параметрами
\(a = 0, b = 1\). Знайти закон розподілу сумарної похибки \(Z = X + Y\).

\textbf{Формулювання:} \(X \sim N(0, 1)\), \(Y \sim U(0, 1)\). Знайти
закон розподілу сумарної похибки \(Z = X + Y\).

\textbf{Розв'язок:}

Щільність розподілу суми двох незалежних випадкових величин знаходиться
через інтеграл згортки:
\[ f_Z(z) = \int_{-\infty}^{\infty} f_X(x) \cdot f_Y(z - x) \,dx \]

Щільності для заданих розподілів: *
\(f_X(x) = \frac{1}{\sqrt{2\pi}} e^{-x^2/2}\) *
\(f_Y(y) = \begin{cases} 1, & y \in [0, 1] \\ 0, & \text{в інших випадках} \end{cases}\)

Підінтегральний вираз не дорівнює нулю тільки тоді, коли
\(0 \le z - x \le 1\), що еквівалентно \(z - 1 \le x \le z\). Це
визначає нові межі інтегрування.

\[ f_Z(z) = \int_{z-1}^{z} \frac{1}{\sqrt{2\pi}} e^{-x^2/2} \cdot 1 \,dx \]

Цей інтеграл виражається через функцію розподілу стандартного
нормального закону
\(\Phi(t) = \int_{-\infty}^{t} \frac{1}{\sqrt{2\pi}} e^{-u^2/2} \,du\).

\textbf{Відповідь:} Щільність розподілу сумарної похибки \(Z\):
\[ f_Z(z) = \Phi(z) - \Phi(z-1) \]

    \subsection{Задача 2}\label{ux437ux430ux434ux430ux447ux430-2}

\textbf{Задача:} Випадкові величини \(X\) та \(Y\) незалежні та обидві
мають рівномірний закон розподілу з параметрами \(a = 0, b = 2\). Знайти
функції розподілу та щільності розподілу випадкової величини
\(Z = X + Y\).

\textbf{Формулювання:} \(X, Y \sim U(0, 2)\) незалежні. Знайти функції
розподілу та щільності розподілу \(Z = X + Y\).

\textbf{Розв'язок:}

Використовуємо згортку. Щільності \(f_X(x) = f_Y(y) = 1/2\) на інтервалі
\([0, 2]\).
\[ f_Z(z) = \int_{-\infty}^{\infty} f_X(x) f_Y(z-x) \,dx = \int \frac{1}{2} \cdot \frac{1}{2} \,dx = \frac{1}{4} \int \,dx \]
Межі інтегрування визначаються перетином умов \(0 \le x \le 2\) та
\(0 \le z-x \le 2\) (тобто \(z-2 \le x \le z\)).

\begin{enumerate}
\def\labelenumi{\arabic{enumi}.}
\tightlist
\item
  \textbf{Для \(0 \le z \le 2\)}: межі інтегрування по \(x\) від \(0\)
  до \(z\). \(f_Z(z) = \int_{0}^{z} \frac{1}{4} \,dx = \frac{z}{4}\).
\item
  \textbf{Для \(2 < z \le 4\)}: межі інтегрування по \(x\) від \(z-2\)
  до \(2\).
  \(f_Z(z) = \int_{z-2}^{2} \frac{1}{4} \,dx = \frac{1}{4}(2 - (z-2)) = \frac{4-z}{4}\).
\end{enumerate}

\textbf{Відповідь (щільність):}
\[ f_Z(z) = \begin{cases} z/4, & 0 \le z \le 2 \\ (4-z)/4, & 2 < z \le 4 \\ 0, & \text{в інших випадках} \end{cases} \]

Інтегруючи щільність, знаходимо функцію розподілу \(F_Z(z)\):

\textbf{Відповідь (функція розподілу):}
\[ F_Z(z) = \begin{cases} 0, & z < 0 \\ z^2/8, & 0 \le z \le 2 \\ 1 - (4-z)^2/8, & 2 < z \le 4 \\ 1, & z > 4 \end{cases} \]

    \subsection{Задача 3}\label{ux437ux430ux434ux430ux447ux430-3}

\textbf{Задача:} Час між запитами до серверу комп'ютерної мережі є
випадковою величиною \(X\), що має експоненціальний закон розподілу з
параметром \(\lambda = 10\). З метою дослідження степені використання
серверу необхідно встановити закон розподілу максимумів випадкової
величини \(X\), тобто деякої випадкової величини \(Z = \max(X)\).

\textbf{Формулювання:} \(X \sim Exp(10)\). Встановити закон розподілу
\(Z = \max(X_1, \dots, X_n)\).

\textbf{Розв'язок:}

Функція розподілу для максимуму з \(n\) незалежних однаково розподілених
випадкових величин знаходиться за формулою: \[ F_Z(z) = [F_X(z)]^n \]

Для експоненціального розподілу \(Exp(\lambda)\) з \(\lambda=10\)
функція розподілу має вигляд:
\[ F_X(z) = \begin{cases} 1 - e^{-10z}, & z \ge 0 \\ 0, & z < 0 \end{cases} \]

Підставляємо \(F_X(z)\) у формулу для \(F_Z(z)\).

\textbf{Відповідь:} Функція розподілу для \(Z = \max(X_1, \dots, X_n)\):
\[ F_Z(z) = \begin{cases} (1 - e^{-10z})^n, & z \ge 0 \\ 0, & z < 0 \end{cases} \]
де \(n\) --- кількість спостережень (розмір вибірки).

    \subsection{Задача 4}\label{ux437ux430ux434ux430ux447ux430-4}

\textbf{Задача:} Випадкова величина \(X\) має рівномірний розподіл з
параметрами \(a = 0, b = \pi\). Знайти функції розподілу та щільності
розподілу випадкової величини \(Z = \sin(X)\), обчислити математичне
сподівання \(M(Z)\) та дисперсію \(D(Z)\).

\textbf{Формулювання:} \(X \sim U(0, \pi)\). Знайти \(F_Z(z)\),
\(f_Z(z)\), \(M(Z)\), \(D(Z)\) для \(Z = \sin(X)\).

\textbf{Розв'язок:}

\begin{enumerate}
\def\labelenumi{\arabic{enumi}.}
\item
  \textbf{Функція та щільність розподілу:} Область значень \(Z\) -
  відрізок \([0, 1]\). Для \(z \in [0, 1]\):
  \(F_Z(z) = P(Z < z) = P(\sin(X) < z)\). На інтервалі
  \(X \in [0, \pi]\) нерівність виконується для
  \(X \in [0, \arcsin(z)) \cup (\pi - \arcsin(z), \pi]\). Оскільки \(X\)
  розподілена рівномірно з щільністю \(f_X(x) = 1/\pi\), ймовірність
  потрапляння в інтервал дорівнює його довжині, поділеній на \(\pi\).
  \(F_Z(z) = \frac{\arcsin(z)}{\pi} + \frac{\pi - (\pi - \arcsin(z))}{\pi} = \frac{2}{\pi}\arcsin(z)\).
  Щільність
  \(f_Z(z) = F'_Z(z) = \frac{d}{dz} \left( \frac{2}{\pi} \arcsin(z) \right) = \frac{2}{\pi\sqrt{1-z^2}}\).
\item
  \textbf{Математичне сподівання \(M(Z)\):}
  \(M(Z) = M(\sin(X)) = \int_{0}^{\pi} \sin(x) \cdot f_X(x) \,dx = \int_{0}^{\pi} \sin(x) \cdot \frac{1}{\pi} \,dx\).
  \(M(Z) = \frac{1}{\pi} [-\cos(x)]_0^\pi = \frac{1}{\pi} (1 - (-1)) = \frac{2}{\pi}\).
\item
  \textbf{Дисперсія \(D(Z) = M(Z^2) - [M(Z)]^2\):}
  \(M(Z^2) = M(\sin^2(X)) = \int_{0}^{\pi} \sin^2(x) \cdot \frac{1}{\pi} \,dx\).
  Використовуємо \(\sin^2(x) = \frac{1 - \cos(2x)}{2}\):
  \(M(Z^2) = \frac{1}{\pi} \int_{0}^{\pi} \frac{1 - \cos(2x)}{2} \,dx = \frac{1}{2\pi} \left[ x - \frac{\sin(2x)}{2} \right]_0^\pi = \frac{1}{2\pi} (\pi) = \frac{1}{2}\).
  \(D(Z) = M(Z^2) - [M(Z)]^2 = \frac{1}{2} - \left(\frac{2}{\pi}\right)^2 = \frac{1}{2} - \frac{4}{\pi^2}\).
\end{enumerate}

\textbf{Відповідь:} * \(F_Z(z) = \frac{2}{\pi} \arcsin(z)\) для
\(z \in [0, 1]\). * \(f_Z(z) = \frac{2}{\pi\sqrt{1-z^2}}\) для
\(z \in [0, 1)\). * \(M(Z) = \frac{2}{\pi}\). *
\(D(Z) = \frac{1}{2} - \frac{4}{\pi^2}\).

    \section{Контрольні
питання}\label{ux43aux43eux43dux442ux440ux43eux43bux44cux43dux456-ux43fux438ux442ux430ux43dux43dux44f}

\begin{center}\rule{0.5\linewidth}{0.5pt}\end{center}

\textbf{1. Як знання закону розподілу значень пікових навантажень у
комп'ютерній мережі підприємства може допомогти при моделюванні та
аналізі пікових навантажень?}

Знання закону розподілу пікових навантажень дозволяє: *
\textbf{Оптимально підбирати обладнання:} Розрахувати необхідну
пропускну здатність для обробки 99.9\% піків, уникаючи зайвих витрат або
недостатньої потужності. * \textbf{Реалістично тестувати мережу:}
Створювати правдоподібні сценарії стрес-тестів для перевірки
відмовостійкості. * \textbf{Ефективно керувати трафіком (QoS):}
Налаштовувати пріоритети для критичних сервісів під час очікуваних
перевантажень. * \textbf{Прогнозувати модернізацію:} Планувати
розширення мережі, аналізуючи, як змінюється розподіл навантажень з
часом.

\begin{center}\rule{0.5\linewidth}{0.5pt}\end{center}

\textbf{2. Як знайти математичне сподівання функції одного випадкового
аргумента?}

Математичне сподівання функції \(Z = g(X)\) знаходиться за ``теоремою
про середнє'', без необхідності обчислювати закон розподілу \(Z\).

\begin{itemize}
\tightlist
\item
  \textbf{Для неперервної ВВ:}
  \(M(g(X)) = \int_{-\infty}^{\infty} g(x) \cdot f_X(x) \,dx\)
\item
  \textbf{Для дискретної ВВ:} \(M(g(X)) = \sum_{i} g(x_i) \cdot p_i\)
\end{itemize}

\begin{center}\rule{0.5\linewidth}{0.5pt}\end{center}

\textbf{3. Як знайти дисперсію функції одного випадкового аргумента?}

Дисперсія функції \(Z = g(X)\) знаходиться за стандартною формулою
\(D(Z) = M(Z^2) - [M(Z)]^2\). Для цього потрібно: 1. Обчислити
\(M(Z) = M(g(X))\). 2. Обчислити \(M(Z^2) = M(g^2(X))\). 3. Підставити
знайдені значення у формулу.

\begin{center}\rule{0.5\linewidth}{0.5pt}\end{center}

\textbf{4. Чому на етапі обчислення закону розподілу функції від
випадкової величини виконати аналіз монотонності функції?}

Аналіз монотонності визначає складність задачі. * \textbf{Якщо функція
монотонна,} рівняння \(g(x) = z\) має \textbf{один} розв'язок
\(x = g^{-1}(z)\), що спрощує розрахунки. * \textbf{Якщо функція
немонотонна,} рівняння \(g(x) = z\) має \textbf{декілька} розв'язків.
При обчисленні ймовірності \(P(Z<z)\) потрібно враховувати всі інтервали
значень \(X\), що задовольняють нерівність, і це ускладнює розв'язання.

\begin{center}\rule{0.5\linewidth}{0.5pt}\end{center}

\textbf{5. Наведіть приклади задач, де виникає потреба в обчисленні
закону розподілу суми випадкових величин.}

\begin{itemize}
\tightlist
\item
  \textbf{Обробка сигналів:} Сума корисного сигналу та випадкового шуму.
\item
  \textbf{Фінансові ризики:} Сумарний збиток портфеля як сума збитків по
  окремих активах.
\item
  \textbf{Теорія надійності:} Загальний час роботи системи, що
  складається з послідовних елементів.
\item
  \textbf{Вимірювальні системи:} Сумарна похибка приладу як сума похибок
  від різних джерел (шум, квантування тощо).
\end{itemize}


    % Add a bibliography block to the postdoc
    
    
    
\end{document}
